\documentclass{article}
%\usepackage{epstopdf}%
%\epstopdfsetup{update}
\usepackage{amsmath} 
\usepackage{amssymb}  
\usepackage{graphicx}
\usepackage{latexsym}
\usepackage{verbatim}
\usepackage{ifthen}
\usepackage{psfrag}
\usepackage{macros/subfigure}
%\include{macros}

% \newboolean{Book}
% \setboolean{Book}{true}
% \newcommand{\IfNotCompilingBook}[1]{\ifthenelse{\boolean{Book}}{}{#1}}
% \newcommand{\chapter}[1]{{\LARGE \bf Chapter 1: \  #1 \\ \vspace{0.5in} }}

% sharp margins
%\setlength{\oddsidemargin}{0.25in}
%\setlength{\evensidemargin}{0.25in}
%\setlength{\textwidth}{6.5in}
%\setlength{\topmargin}{-0.25in}
%\setlength{\textheight}{9in}

\newcounter{chapter}
\setcounter{chapter}{1}
\newcommand{\chapter}[1]{{\LARGE \center \bf #1 \\ \vspace{0.5in} }}
%%%%%%%%%%%%%%%%%%%%%%%%%%%%%%%%%%%%%%%%%%%%%%%%%%%%%%%%%%%%%%%%%%%%%%%%%
%%%%%%%%%%%%%%%%%%%%%%%%%%%%%%%%%%%%%%%%%%%%%%%%%%%%%%%%%%%%%%%%%%%%%%%%%
%%%%%%%%%%%%%%%%%%%%%%%%%%%%%%%%%%%%%%%%%%%%%%%%%%%%%%%%%%%%%%%%%%%%%%%%%
%%%%%  THEOREMS AND ENVIRONMENTS
%%%%%%%%%%%%%%%%%%%%%%%%%%%%%%%%%%%%%%%%%%%%%%%%%%%%%%%%%%%%%%%%%%%%%%%%%

\newtheorem{nntheorem}{\bf Theorem}[chapter]
\newtheorem{nnlemma}[nntheorem]{\bf Lemma}
\newtheorem{nndefinition}[nntheorem]{\bf Definition}
\newtheorem{nncorollary}[nntheorem]{\bf Corollary}
\newtheorem{nnproposition}[nntheorem]{\bf Proposition}
\newtheorem{nnassumption}[nntheorem]{\bf Assumption}
\newtheorem{nexample}[nntheorem]{\bf Example}
\newtheorem{nnremark}[nntheorem]{\bf Remark}
\newtheorem{nnproblem}[nntheorem]{\bf Exercise}

\newenvironment{theorem}[1]
{\begin{nntheorem}{\rm\textrm{(#1)}}\sl}
{\end{nntheorem}}

\newenvironment{proposition}[1]
{\begin{nnproposition}{\rm\textrm{(#1)}}\sl}
{\end{nnproposition}}

\newenvironment{lemma}[1]
{\begin{nnlemma}{\rm\textrm{(#1)}}\sl}
{\end{nnlemma}}

\newenvironment{corollary}[1]
{\begin{nncorollary}{\rm\textrm{(#1)}}\sl}
{\end{nncorollary}}

\newenvironment{definition}[1]
{\begin{nndefinition}{\rm\textrm{(#1)}}\sl}
{\end{nndefinition}}

\newenvironment{assumption}[1]
{\begin{nnassumption}{\rm\textrm{(#1)}}\sl}
{\end{nnassumption}}


\newenvironment{remark}[1]
{\begin{nnremark}{\rm\textrm{(#1)}}\sl}
{\end{nnremark}}

\newenvironment{problem}[1]
{\begin{nnproblem}{\rm\textrm{(#1)}}\sl}
{\end{nnproblem}}


%%%%%%%%%%%%%%%%%%%%%%%%%%%%%%%%%%%%%%%%%%%%%%%%%%%%%%%%%%%%%%%%%%%%%%%%%

\newcommand{\eoe}
           {\hspace*{\fill}{$\vcenter{\hrule height1pt 
                     \hbox{\vrule width1pt height3pt 
            \kern3pt \vrule width1pt} \hrule height1pt}$} }

\newenvironment{example}[1]
{\begin{nexample}{\rm\textrm{(#1)}}\rm}{\eoe\end{nexample}}

%%%%%%%%%%%%%%%%%%%%%%%%%%%%%%%%%%%%%%%%%%%%%%%%%%%%%%%%%%%%%%%%%%%%%%%%%

\newcommand{\eop}
           {\hspace*{\fill}{$\vcenter{\hrule height1pt 
                     \hbox{\vrule width1pt height5pt 
            \kern5pt \vrule width1pt} \hrule height1pt}$} }

\newenvironment{proof}
{\par\noindent\textbf{Proof.}}{\eop\smallskip\vskip 3 pt}


%%%%%%%%%%%%%%%%%%%%%%%%%%%%%%%%%%%%%%%%%%%%%%%%%%%%%%%%%%%%%%%%%%%%%%%%%
%%%%%%%%%%%%%%%%%%%%%%%%%%%%%%%%%%%%%%%%%%%%%%%%%%%%%%%%%%%%%%%%%%%%%%%%%
%%%%%%%%%%%%%%%%%%%%%%%%%%%%%%%%%%%%%%%%%%%%%%%%%%%%%%%%%%%%%%%%%%%%%%%%%
%%%%%  PRETTY OBVIOUS NEWCOMMANDS
%%%%%%%%%%%%%%%%%%%%%%%%%%%%%%%%%%%%%%%%%%%%%%%%%%%%%%%%%%%%%%%%%%%%%%%%%
 
\newcommand{\ball}{{\mathbb B}}
%\newcommand{\con}{{\mathop{\rm con}\nolimits}}
%\newcommand{\clcon}{{\overline{\con}}}
\newcommand{\dom}{\mathop{\rm dom}\nolimits}
\newcommand{\glim}{\mathop{\rm gph\mbox{-}lim}}
\newcommand{\glimsup}{\mathop{\rm gph\mbox{-}lim\,sup}}
\newcommand{\gliminf}{\mathop{\rm gph\mbox{-}lim\,inf}}
\newcommand{\gph}{\mathop{\rm gph}\nolimits}
\renewcommand{\iint}{\mathop{\rm int}\nolimits}
\newcommand{\integers}{{\mathbb Z}}
\newcommand{\iti}{{i\to\infty}}
\newcommand{\kti}{{k\to\infty}}
\newcommand{\KL}{{{\mathcal{K}\mathcal{L}}}}
\newcommand{\KLL}{{{\mathcal{K}\mathcal{L}\mathcal{L}}}}
\newcommand{\naturals}{{\mathbb N}}
\newcommand{\ox}{{\bar{x}}}
\newcommand{\reals}{{\mathbb R}}
\renewcommand{\Re}{{\mathbb R}}
\newcommand{\realsplus}{{\reals_{\geq 0}}}
\newcommand{\realspplus}{{\reals_{>0}}}
\newcommand{\rge}{\mathop{\rm rge}\nolimits}
%\newcommand{\rge}{\mathop{\rm rge}}      
%\newcommand{\rge}{{\mathop{\rm rge}\nolimits}}


% \newcommand{\tto}{\;{\lower 1pt \hbox{$\rightarrow$}}\kern -10pt
%            \hbox{\raise 2pt \hbox{$\rightarrow$}}\;}

\newcommand{\tto}{\;{\lower 1pt \hbox{$\rightarrow$}}\kern -12pt
           \hbox{\raise 2pt \hbox{$\rightarrow$}}\;}



%%%%%%%%%%%%%%%%%%%%%%%%%%%%%%%%%%%%%%%%%%%%%%%%%%%%%%%%%%%%%%%%%%%%%%%%%
%%%%%%%%%%%%%%%%%%%%%%%%%%%%%%%%%%%%%%%%%%%%%%%%%%%%%%%%%%%%%%%%%%%%%%%%%
%%%%%%%%%%%%%%%%%%%%%%%%%%%%%%%%%%%%%%%%%%%%%%%%%%%%%%%%%%%%%%%%%%%%%%%%%
%%%%%  NEWCOMMANDS TO ARGUE ABOUT
%%%%%%%%%%%%%%%%%%%%%%%%%%%%%%%%%%%%%%%%%%%%%%%%%%%%%%%%%%%%%%%%%%%%%%%%%

%% compact attractor
\newcommand{\A}{\mathcal{A}}
%% basin of attraction of a compact attractor
\newcommand{\BA}{\mathcal{B}_\A}
%% generic measuement error
\newcommand{\e}{e}
%% hybrid system
\newcommand{\HS}{\mathcal{H}}
%% hybrid system with data
\newcommand{\HSdata}{\HS=(O,F,C,G,D)}
%% hybrid system data only
\newcommand{\data}{(O,F,C,G,D)}
%% hybrid system with data (lower case)
\newcommand{\datal}{(O,f,C,g,D)}
%% hybrid system regularized data only
\newcommand{\regdata}{(O,\reg{F},\reg{C},\reg{G},\reg{D})}
%% hybrid system with measurement error
\newcommand{\HSe}{{\HS_e}}
%% hybrid system, regularized
\newcommand{\HSreg}{{\reg{\HS}}}
%% generic hybrid time domain
\newcommand{\htd}{E}
%% indicator of A 
\newcommand{\indi}{\omega}
%% generic compact set
\newcommand{\K}{K}
%% generic KL function
\newcommand{\kl}{\gamma}
%% length of a hybrid time domain
\newcommand{\length}{\mathop{\rm length}\nolimits}
%% generic set valued mapping
\newcommand{\map}{M}
%% state space
\renewcommand{\O}{O}
%% pre basin of attraction 
\newcommand{\preBA}{{\BA^p}}
%% admissible radius of perturbation
\newcommand{\rad}{\rho}
%% reachable set
%\newcommand{\reach}{\mathcal{R}}
%% regularization of C D F G
\newcommand{\reg}[1]{{\widehat{#1}}}
%% saturation function
\newcommand{\sat}{{\rm sat}}
%% sequence, for example \seq{x}{i} produces \{x_i\}_{i=0}^\infty
\newcommand{\seq}[2]{{\{#1_{#2}\}_{#2=1}^\infty}}
%% subsequence, for example \seq{x}{i}{k} produces \{x_{i_k}\}_{i=0}^\infty
\newcommand{\subseq}[3]{{\{#1_{#2_{#3}}\}_{#3=1}^\infty}}
%% generic set
\newcommand{\set}{S}
%% solution to a hybrid system or just a hybrid arc
\newcommand{\sol}{x}
%% solution to a hybrid closed-loop system (control)
\newcommand{\solcl}{\zeta}
%%
\newcommand{\solcldot}{\dot{\zeta}}
%%
\newcommand{\solclplus}{\zeta^+}
%% initial point for solution to a hybrid system 
\newcommand{\solinit}{\xi}
%% set of maximal solutions to a hybrid system 
\newcommand{\So}{{\mathcal{S}}}
%% set of maximal solutions to a hybrid system \HS
\newcommand{\Sol}{{\mathcal{S}_\HS}}
%% continuous time arc
\newcommand{\solc}{z}
%%
\newcommand{\solcdot}{{\dot{\solc}}}
%%
\newcommand{\soldot}{{\dot{\sol}}}
%% discrete time arc
\newcommand{\sold}{z}
%%
\newcommand{\soldplus}{\sold^+}
%%
\newcommand{\solplus}{{\sol^+}}
%% symbol for switching system
\renewcommand{\SS}{\Sigma}
%\renewcommand{\SS}{\mathcal{S}}
%% supremum in time of a hybrid time domain
\newcommand{\supt}{{\sup\nolimits_t}} 
%% supremum in jumps of a hybrid time domain
\newcommand{\supj}{{\sup\nolimits_j}}
%% generic open set 
\newcommand{\U}{\mathcal{U}}
%% text referring to the type of document is being compiled
\newcommand{\book}{thesis}
% set
% \newcommand{\bigbrace}[1]{\left\{#1\right\}}
% \newcommand{\set}[2]{\bigbrace{#1\ \left| \ #2 \right.}}
% \newcommand{\setsmall}[2]{\{#1\ | \ #2 \}}

%% j-th interval for continuous time t
\newcommand{\intj}{I^j}
%% J-th interval for continuous time t
\newcommand{\intJ}{I^J}
%% 0-th interval for continuous time t
\newcommand{\intzero}{I^0}
%% shorthand for varepsilon
\newcommand{\eps}{\varepsilon}
%% shorthand for \overline
\newcommand{\ol}[1]{\overline{#1}}
%% matrix 
\newcommand{\matt}[1]{\begin{bmatrix}#1\end{bmatrix}} 
%% set definition
\newcommand{\bigbrace}[1]{\left\{#1\right\}}
\newcommand{\defset}[2]{\bigbrace{#1\ \left| \ #2 \right.}}
% sign function
\newcommand{\sign}{{\mathop{\rm sign}\nolimits}}
% floor function
\newcommand{\floor}{{\mathop{\rm floor}\nolimits}}
%
\newcommand{\HBC}{hybrid basic conditions}

% List of macros that are not in macrosbook.tex and come from rgsMacros.sty 
%% function for plant flow dynamics
\newcommand{\fp}{f_P}
%% function for controller flow dynamics
\newcommand{\fc}{f_c}
%% function for noise
\newcommand{\fe}{m}
%% function for controller jump dynamics
\newcommand{\gc}{G_c}
%% function for controller jump dynamics - continuout part
\newcommand{\gccont}{G^c_c}
%% function for controller jump dynamics - discrete part
\newcommand{\gcdisc}{G^d_c}
%% plant state
\newcommand{\xp}{x}
%%
\newcommand{\xpdot}{\dot{x}}
%%
\newcommand{\xpplus}{\xp^+}
%% controller state
\newcommand{\xc}{x_c}
%%
\newcommand{\xcdot}{\dot{x}_c}
%%
\newcommand{\xcplus}{x^+_c}
%%
\newcommand{\xccont}{\xi}
%%
\newcommand{\xccontdot}{\dot{\xccont}}
%%
\newcommand{\xccontplus}{\xccont^+}
%%
\newcommand{\xcdisc}{q}
%%
\newcommand{\xcdiscdot}{\dot{\xcdisc}}
%%
\newcommand{\xcdiscplus}{\xcdisc^+}
%% controller output
\newcommand{\kc}{\kappa}
%% filter state
\newcommand{\xf}{x_f}
%%
\newcommand{\xfdot}{\dot{x}_f}
%%
\newcommand{\xfplus}{\xf^+}
%% filter state sensor
\newcommand{\xs}{x_{s}}
%%
\newcommand{\xsdot}{\dot{x}_{s}}
%%
\newcommand{\xsplus}{\xs^+}
%% filter state actuator
\newcommand{\xa}{x_{a}}
%%
\newcommand{\xadot}{\dot{x}_{a}}
%%
\newcommand{\xaplus}{\xa^+}
%% filter state input
\newcommand{\xu}{x_{u}}
%%
\newcommand{\xudot}{\dot{x}_{u}}
%%
\newcommand{\xuplus}{\xu^+}
%% filter constant
\newcommand{\epsf}{\varepsilon_f}
%% filter sensor/actuator constant
\newcommand{\epsd}{\varepsilon_d}
%% filter control smoothing
\newcommand{\epsu}{\varepsilon_u}
%% plant state domain
\newcommand{\xpdomain}{\reals^{n_p}}
%% controller state domain
\newcommand{\xcdomain}{\reals^{n_c}}
%% filter state domain
\newcommand{\xfdomain}{\reals^{n_f}}
%% filter sensor state domain
\newcommand{\xsdomain}{\reals^{n_s}}
%% filter actuator state domain
\newcommand{\xadomain}{\reals^{n_a}}
%% time state domain
\newcommand{\timerdomain}{\reals}
%% continuous state of controller  domain
\newcommand{\xccontdomain}{\reals^{n_c-1}}
%% discrete state of controller  domain
\newcommand{\xcdiscdomain}{Q}
%% sol domain
\newcommand{\soldomain}{\reals^{n}}
%% solcl domain
\newcommand{\solcldomain}{\xpdomain\times\xcdomain}
%% input domain
\newcommand{\udomain}{\reals^{m}}
%% controller shorthand notation
\newcommand{\HScdata}{(\O,\fc,\Cc,\gc,\Dc,\kc)}
%% closed-loop system 
\newcommand{\HScl}{\HS_{cl}}      
%% closed-loop system with filtered measurement noise
\newcommand{\HSfcl}{\HS^{\epsf}_{cl}}      
%% closed-loop system with sensor and actuator dynamics
\newcommand{\HSpcl}{\HS^{\epsd}_{cl}}      
%% closed-loop system with control smoothing
\newcommand{\HSucl}{\HS^{\epsu}_{cl}}      
%% closed-loop system with sample and hold
\newcommand{\HSclSH}{\HS_{cl}^{S/H}}      
%% hybrid controller
\newcommand{\HSc}{\HS_{c}}      
%% hybrid system with measurement noise
%\newcommand{\HSe}{\HS^e}      
%% end markers
\newcommand{\myendbox}{\null \hfill $\Box$}
\newcommand{\myendtriangle}{\null \hfill $\triangle$}
%% end marker for assumptions
%\newcommand{\endassume}{\myendbox}
%% end marker for examples
%\newcommand{\endex}{\myendtriangle}
%% controller flow set
\newcommand{\Cc}{C_c}
%% controller jump set
\newcommand{\Dc}{D_c}
%% matrix
%\newcommand{\matt}[1]{\begin{bmatrix}#1\end{bmatrix}} 
%% big delimiters
\newcommand{\bigpar}[1]{\left(#1\right)}
\newcommand{\bigbracket}[1]{\left[#1\right]}
%\newcommand{\bigbrace}[1]{\left\{#1\right\}}
\newcommand{\bigbar}[1]{\left| #1\right|}
%% big set
\newcommand{\bigset}[2]{\bigbrace{#1\ \left| \ #2 \right.}}
%% timer variable
\newcommand{\timer}{\tau}
%% timer variable dot
\newcommand{\timerdot}{\dot{\tau}}
%% timer variable plus
\newcommand{\timerplus}{\timer^+}
%% timer parameter
\newcommand{\timerpar}{\ol{\tau}}
%% timer state for sampler
\newcommand{\timers}{\tau_s}
%%
\newcommand{\timersdot}{\dot{\tau}_s}
%%
\newcommand{\timersplus}{\timers^+}
%% timer constant for sampler
\newcommand{\timerscons}{T_s}
%% timer state for holder
\newcommand{\timerc}{\tau_c}
%%
\newcommand{\timercdot}{\dot{\tau}_c}
%%
\newcommand{\timercplus}{\timerc^+}
%% timer constant for holder
\newcommand{\timerccons}{T_c}
%% state for sampler
\newcommand{\zs}{z_s}
%%
\newcommand{\zsdot}{\dot{z}_s}
%%
\newcommand{\zsplus}{\zs^+}
%% state for holder
\newcommand{\zc}{z_c}
%%
\newcommand{\zcdot}{\dot{z}_c}
%%
\newcommand{\zcplus}{\zc^+}
%% state for memory for computations
\newcommand{\zm}{z_m}
%%
\newcommand{\zmdot}{\dot{z}_m}
%%
\newcommand{\zmplus}{\zm^+}
%% domain for sampler timer 
\newcommand{\timersdomain}{\reals}
%% domain for sampler state 
\newcommand{\zsdomain}{\xpdomain}
%% domain for holder timer 
\newcommand{\timercdomain}{\reals}
%% domain for holder state 
\newcommand{\zcdomain}{\xcdomain}
%% domain for the memory state
\newcommand{\zmdomain}{\xpdomain}
%% \non
\newcommand{\non}{\nonumber}
%% \classKinfnty
\newcommand{\classKinfty}{{\mathcal{K}}_{\infty}}
%% donut
\newcommand{\donutthree}{\Omega_{\A}(0,\Delta_s)}
%% \R
\newcommand{\R}{{\mathcal{R}}}
%%
\newcommand{\M}{{\mathcal{M}}}
%%
\newcommand{\realsgeq}{{\reals_{\geq 0}}}
%%
\newcommand{\nats}{\mathbb{N}}      
%%
\renewcommand{\K}{K}
%%
\newcommand{\X}{{\mathcal{X}}}
%%
\newcommand{\mattarraytwo}[1]{\begin{array}{ll}#1\end{array}} 
%%
\newcommand{\mattarrayone}[1]{\begin{array}{l}#1\end{array}} 
%%
%\newcommand{\BAn}[1]{{\mathcal B}_{\A_{#1}}} 
%%
\newcommand{\Qcatch}{Q_j^c}
\newcommand{\Qthrow}{Q_j^t}
%%
\newcommand{\B}{{\mathcal B}} 
% special commands
\newcommand{\Au}{\A_{u}} 
\newcommand{\Ar}{\A_{r}} 
\newcommand{\Aur}{\A_{ur}} 
\newcommand{\Aru}{\A_{ru}} 
\newcommand{\HSr}{\HS_r} 
%%
\newcommand{\zdot}{\dot{z}}     
%%
\newcommand{\omegaset}{\Omega_{\HS}}
%%
%\newcommand{\non}{\nonumber}
%%
\newcommand{\eqn}[1]{\begin{eqnarray}#1\end{eqnarray}} 
%%
\newcommand{\J}{{\mathcal{J}}}
%%
\newcommand{\ReNone}{{\mathbb{R}^{n}}}
\newcommand{\ReNtwo}{{\mathbb{R}^{n_c}}}
\newcommand{\ReMone}{{\mathbb{R}^{m}}}
\newcommand{\ReMtwo}{{\mathbb{R}^{m_c}}}
%%
\newcommand{\HSgensystem}{\HSdata} 
\newcommand{\HSsimsystem}{\HS_s=(O,F_s,C_s,G_s,D_s)} 
\newcommand{\HSs}{\HS_s} 
%%
\newcommand{\graph}{\mbox{gph }}
%% 
\newcommand{\HScontdata}{(O,f,C,\emptyset,\emptyset)}
%% 
\newcommand{\HSdiscdata}{(O,\emptyset,\emptyset,g,D)}
%%
\newcommand{\HSsystemnO}{\HS=(f,C,g,D,\reals^n)}   
%%
\newcommand{\RH}{\R_{\HS}}
%
\newcommand{\Ball}{\ball}
%
%\newcommand{\ox}{\bar{x}}      
\newcommand{\oz}{\bar{z}}
\newcommand{\ot}{\bar{t}}      
\newcommand{\oj}{\bar{j}}      
%%
\newcommand{\co}{\mathop{\rm co}}     
\newcommand{\cco}{\overline{\mathop{\rm co}}}
%%
\newcommand{\classKLL}{\KLL}
%%
\newcommand{\xdot}{\dot{x}}
%%
\newcommand{\ve}{\varepsilon}  
\newcommand{\ydot}{\dot{y}}   
%\newcommand{\zdot}{\dot{z}}   
\newcommand{\I}{\mathcal{I}}  
\newcommand{\varphidot}{\dot{\varphi}}   
\newcommand{\psidot}{\dot{\psi}}   
\newcommand{\HSsystem}{\HS=(O,f,C,g,D)}   
%%
\newcommand{\tausdot}{\dot{\tau}_s}
\newcommand{\taucdot}{\dot{\tau}_c}
\newcommand{\zthdot}{\dot{\widetilde{z}}_h}
% %\newcommand{\zsdot}{\dot{z}_s}
% %\newcommand{\zhdot}{\dot{z}_h}
% %\newcommand{\zmdot}{\dot{z}_m}
\newcommand{\tausplus}{{\tau}_s^+}
\newcommand{\taucplus}{{\tau}_c^+}
\newcommand{\zthplus}{{\widetilde{z}}_h^+}
% %\newcommand{\zsplus}{{z}_s^+}
% %\newcommand{\zhplus}{{z}_h^+}
% %\newcommand{\zmplus}{{z}_m^+}
\newcommand{\taus}{{\tau}_s}
\newcommand{\tauc}{{\tau}_c}
\newcommand{\zth}{{\widetilde{z}}_h}
\newcommand{\eone}{\zth - \zh}
\newcommand{\etwo}{\zs - x}
\newcommand{\ethree}{x-\zm}
\newcommand{\donut}{\Omega_{\A}(\delta_s,\Delta_s)}
\newcommand{\donuttwo}{\Omega_{\A}(0,\delta_s)}
% %\newcommand{\donutthree}{\Omega_{\A}(0,\Delta_s)}
\newcommand{\donutfour}{\Omega_{\A}(0,\Delta_s')}
\newcommand{\donutfive}{\Omega_{\A}(0,\delta_s')}
%%
\newcommand{\C}{\mathcal K}
%%
\newcommand{\inftynorm}[1]{\left|#1\right|_{\infty}} 
%%
\newcommand{\Mfull}{(\M+\eps\ball)\cap \C}
\newcommand{\Mfullone}{(\M+\eps_1\ball)\cap \C}
\newcommand{\Mepsone}{\M+\eps_1\ball}
\newcommand{\Meps}{\M+\eps\ball}
%%
\newcommand{\BAt}{{\mathcal B}_{\tilde{\A}}} 
\newcommand{\BAn}[1]{{\mathcal B}_{\A_{#1}}} 
% \newcommand{\Au}{\A_{u}} 
% \newcommand{\Ar}{\A_{r}} 
% \newcommand{\Aur}{\A_{ur}} 
% \newcommand{\Aru}{\A_{ru}} 
% \newcommand{\Qcatch}{Q_j^c}
% \newcommand{\Qthrow}{Q_j^t}
\newcommand{\cball}{\ol{\mathbb{B}}}      
\newcommand{\dist}{{\rm dist}}
\newcommand{\reach}{\mathop{\rm reach}\nolimits}



%%% MACROS FOR HCBOOK

%% plant's state
%\newcommand{\xp}{z}
%%
%\newcommand{\xpdot}{\dot{\xp}}
%%
\newcommand{\xpddot}{\ddot{\xp}}
%% plant's input
\newcommand{\up}{u}
%% plant's output
\newcommand{\yp}{y}
%% plant's right-hand side
%\newcommand{\fp}{F_P}
\newcommand{\gp}{G_P}
\newcommand{\Cp}{C_P}
\newcommand{\Dp}{D_P}
%% plant's output function
\newcommand{\hp}{h}
\newcommand{\upSpace}{{\cal U}}


%% controller's state
%\newcommand{\xc}{\eta}
%%
%\newcommand{\xcdot}{\dot{\xc}}
%%
\newcommand{\xcddot}{\ddot{\xc}}
%% controller's input
\newcommand{\uc}{v}
%% controller's output
\newcommand{\yc}{\zeta}
%% controller's right-hand side
%\newcommand{\fc}{F_C}
%\newcommand{\gc}{G_C}
%% controller's output function
\newcommand{\hc}{\kappa}
%\newcommand{\Cc}{C_C}
%\newcommand{\Dc}{D_C}


%% interface's state
\newcommand{\xn}{\lambda}
%%
\newcommand{\xndot}{\dot{\xn}}
%%
\newcommand{\xnddot}{\ddot{\xn}}
%% controller's input
\newcommand{\un}{w}
%% controller's output
\newcommand{\yn}{\psi}
%% controller's right-hand side
\newcommand{\fn}{F_I}
\newcommand{\gn}{G_I}
%% controller's output function
\newcommand{\hn}{\varphi}
\newcommand{\Cn}{C_I}
\newcommand{\Dn}{D_I}

%% logic state
\newcommand{\xlogic}{q}
\newcommand{\xlogicdot}{\dot{\xlogic}}
\newcommand{\xlogicSpace}{Q}
\newcommand{\xcSpace}{\Upsilon}
\newcommand{\ucSpace}{{\cal V}}

%% timer state
\newcommand{\xtimer}{\tau}
\newcommand{\xtimerdot}{\dot{\xtimer}}
\newcommand{\xtimerp}{\tau^*}
\newcommand{\xtimerSpace}{[0,\xtimerp]}

%% memory state
\newcommand{\xmem}{\ell}
\newcommand{\xmemdot}{\dot{\xmem}}

%% memory state
\newcommand{\x}{x}
%\newcommand{\xdot}{\dot{\x}}
%\newcommand{\X}{{\cal X}}

% cco
%\newcommand{\cco}{\overline{\mathop{\rm co}}}

% LTL
\newcommand{\True}{{\tt True}\xspace}
\newcommand{\False}{{\tt False}\xspace}
\newcommand{\fa}{\mathfrak{f}_1}
\newcommand{\fb}{\mathfrak{f}_2}
\newcommand{\p}{\mathfrak{a}}
\newcommand{\pa}{\mathfrak{a}_1}
\newcommand{\pb}{\mathfrak{a}_2}

%%% Highlight modifications
\def\startmodif{\color{blue}}
\def\stopmodif{\color{black}\normalcolor}



%%% Comments
%\newcommand{\rafal}[1]{{\noindent \color{red} \tt \small #1}}
%\newcommand{\andy}[1]{{\noindent \color{green} \tt \small #1}}
%\newcommand{\ricardo}[1]{{\noindent \color{blue} \tt \small #1}}
%\def\startmodif{\color{red}}
%\def\stopmodif{\color{black}\normalcolor}
%
%\def\startmodifa{\color{green}}
%\def\stopmodifa{\color{black}\normalcolor}
%
%\def\startmodifaa{\color{black}}
%\def\stopmodifaa{\color{black}\normalcolor}
%
%\def\startmodifr{\color{blue}}
%\def\stopmodifr{\color{black}\normalcolor}
%
%\def\startmodifra{\color{black}}
%\def\stopmodifra{\color{black}\normalcolor}



%\newcommand{\Sphere}{\mathbb{S}}

%\newcommand{\T}{\mathcal{T}}

%\newcommand{\realsgeq}{{\reals_{\geq 0}}}
%\newcommand{\nats}{\mathbb{N}}
\newcommand{\SSS}{E}
%\newcommand{\non}{\nonumber}
\def\ba{\begin{array}}
\def\ea{\end{array}}
%\newcommand{\eqn}[1]{\begin{eqnarray}#1\end{eqnarray}} 
\newcommand{\Ir}{{\cal I}(r)}
\newcommand{\Irzero}{{\cal I}(0)}
\def\uflow{u_c} %v_c
\def\uflowTwo{\tilde{u}_c} %v_c
\def\uflowmap{u_c} % v_c
\def\uflowset{u_c} % v_c

\def\ujump{u_d} % u_d
\def\ujumpTwo{\tilde{u}_d} % u_d
\def\ujumpmap{u_d} % u_d
\def\ujumpset{u_d} % u_d

\def\yflow{y_c}
\def\yflowmap{y}
\def\yflowset{y}

\def\yjump{y_d}
\def\yjumpmap{y}
\def\yjumpset{y}
\def\restcoefficentBall{e_c}
%OLD
                                          
%%% RICARDO ADDITIONS
\usepackage{color}
\newtheorem{helptheorem}{Theorem}[section]

\newtheorem{helplemma}[helptheorem]{Lemma}

\newtheorem{helpcorollary}[helptheorem]{Corollary}

\newtheorem{helpexample}[helptheorem]{Example}

\newtheorem{helpproposition}[helptheorem]{Proposition}

\newtheorem{helpremark}[helptheorem]{Remark}

\newtheorem{helpdefinition}[helptheorem]{Definition}

\newtheorem{helpassumption}[helptheorem]{Assumption}

\newtheorem{helpstassumption}[helptheorem]{Standing Assumption}

\newcommand{\HyEQfolder}{\mbox{{\ttfamily{ \color{red} TBD }}}}
\newcommand{\HyEQversion}{\mbox{v3.00}}
\newcommand{\matlab}{\textsc{Matlab}}

\newcommand{\ricardo}[1]{{\color{blue} #1}}
\newcommand{\pn}[1]{{\color{red} #1}}

\usepackage[left=1in,top=1in,right=1in,bottom=1in,nohead]{geometry}
%\usepackage{subfigure}
\usepackage{psfrag}
\usepackage{pstool}
\usepackage{url}
\geometry{letterpaper}
\usepackage{fancyvrb} 
\usepackage{listings}
\usepackage{hyperref}
\usepackage{framed}

%%%%%%%%%%%%%%%%%%%%%%%%%%%%%%%%%%%%%%%%%%%%%%%%%%%%%%%%%%%%%%%%%%%%%%
%%% VERSION CONTROL COMMAND
%%% For Conference, Change to "true"
%%% For Report, Change to "false"
\usepackage{ifthen}
\newboolean{StandAloneExample}
\setboolean{StandAloneExample}{false}
%\newcommand{\NotSAE}[1]{\ifthenelse{\boolean{StandAloneExample}}{}{{\color{red}#1}}}
%\newcommand{\IfSAE}[2]{\ifthenelse{\boolean{StandAloneExample}}{#1}{{\color{blue}#2}}}   
\newcommand{\NotSAE}[1]{\ifthenelse{\boolean{StandAloneExample}}{}{#1}}
\newcommand{\IfSAE}[2]{\ifthenelse{\boolean{StandAloneExample}}{#1}{#2}}
%%%%%%%%%%%%%%%%%%%%%%%%%%%%%%%%%%%%%%%%%%%%%%%%%%%%%%%%%%%%%%%%%%%%%%%%%

\newcommand{\thetoolbox}{the Hybrid Equations Toolbox}
\newcommand{\thetoolboxfile}{the Hybrid Equations Toolbox}
\newcommand{\needsupdate}[1]{{\color{red} [Needs update:] #1}}
\newcommand{\todo}[1]{{\color{red} [To do:] #1}}

%%%%%%%%%%%%%%%%%%

\begin{document}

\chapter{Hybrid Equations (HyEQ) Toolbox\\
{\it For Simulating Hybrid Systems in 
MATLAB and Simulink$^{\textup{\footnotesize\textregistered}}$}
}
\label{app:simulations}

\vspace{-0.4in}

\noindent\makebox[\textwidth][c]{% Make minipage centered.
\begin{minipage}[t]{1.10\textwidth}
  \begin{minipage}[t]{0.23\textwidth}
  \begin{center}
  Ricardo G. Sanfelice\\%ricardo@ucsc.edu
  {\it University of California \\ Santa Cruz, CA 95064}\\
  {\it USA}
  \end{center}
  \end{minipage}
  \hfill
  \begin{minipage}[t]{0.23\textwidth}
  \begin{center}
  David A. Copp\\%dacopp@engr.ucsb.edu
  {\it University of California \\ Santa Barbara, CA 93109}\\
  {\it USA}
  \end{center}
  \end{minipage}
  \hfill
  \begin{minipage}[t]{0.23\textwidth}
  \begin{center}
  Pablo Nanez\\%pa.nanez49@uniandes.edu.co
  {\it Universidad de Los Andes}\\
  {\it Colombia}
  \end{center}
  \end{minipage}
  \hfill
  \begin{minipage}[t]{0.23\textwidth}
  \begin{center}
  Paul K. Wintz\\
  {\it University of California \\ Santa Cruz, CA 95064}\\
  {\it USA}
  \end{center}
  \end{minipage}%
\end{minipage}%
}

\begin{center}
{\today}
\end{center}
%\bigskip

%%%%%%%%%%%%%%%%%%%%%%%%%%%%%%%%%

\begin{abstract}
This note describes the Hybrid Equations (HyEQ) Toolbox implemented in 
\matlab{} and Simulink for the simulation of hybrid dynamical systems. 
This toolbox is capable of simulating individual and interconnected 
hybrid systems where multiple hybrid systems are connected and interact 
such as a bouncing ball on a moving platform, fireflies synchronizing their flashing, and more. 
The Simulink implementation includes four basic blocks that define the dynamics of a hybrid system. 
These include a flow map, flow set, jump map, and jump set. The flows and jumps of the system are 
computed by the integrator system which is comprised of blocks that compute the continuous dynamics 
of the hybrid system, trigger jumps, update the state of the system and simulation time at jumps, 
and stop the simulation. 

\section{MATLAB-based Hybrid Equation Simulator}
\label{sec:litesolver}
One way to simulate hybrid systems is to use 
ODE function calls with events in \matlab{}. 
Such an implementation gives fast simulation of a hybrid system.

\needsupdate{
In the \matlab{}-based HyEQ solver, four basic functions are 
used to define the {\em data} of the hybrid system $\HS$
as in \eqref{eqn:HS} (without inputs):
\begin{itemize}
\item The flow map is defined in the \matlab{} function {\tt
f.m}. The input to this function is a vector with components 
defining the state of the system $x$. Its output is the value of the flow map $f$.
\item The flow set is defined in the \matlab{} function {\tt
C.m}. The input to this function is a vector with components 
defining the state of the system $x$. Its output is equal to $1$ 
if the state belongs to the set $C$ or equal to $0$ otherwise.
\item The jump map is defined in the \matlab{} function {\tt
g.m}. Its input is a vector with components defining the state of the system $x$. 
Its output is the value of the jump map $g$.
\item The jump set is defined in the \matlab{} function {\tt
D.m}. Its input is a vector with components defining the state of the system $x$. 
Its output is equal to $1$ if the state belongs to $D$ or equal to $0$ otherwise.
\end{itemize}
}

Our \matlab{}-based HyEQ Simulator uses a main function {\tt run.m} to initialize, run, 
and plot solutions for the simulation, functions {\tt f.m, C.m, g.m,} and {\tt D.m} 
to implement the data of the hybrid system, and {\tt HyEQsolver.m} 
which will solve the differential equations by integrating the continuous dynamics, 
$\dot{x}=f(x)$, and jumping by the update law $x^+ = g(x)$. 
The ODE solver called in {\tt HyEQsolver.m} initially uses the initial or most recent step size, 
and after each integration, the algorithms in {\tt HyEQsolver.m} check to see 
if the solution is in the set $C$, $D$, or neither. 
Depending on which set the solution is in, 
the simulation is accordingly reset following the dynamics given in $f$ or $g$, 
or the simulation is stopped. 
This implementation is fast because it also does not store variables to the workspace 
and only uses built-in ODE function calls.

Time and jump horizons are set for the simulation using {\tt TSPAN = [TSTART TFINAL]} 
as the time interval of the simulation and {\tt JSPAN = [JSTART JSTOP]} 
as the interval for the number of discrete jumps allowed. 
The simulation stops when either the time or jump horizon, 
i.e., the final value of either interval, 
is reached.

The example below shows how to use the HyEQ solver to simulate a bouncing ball.

\begin{example}{Bouncing ball with Lite HyEQ Solver}
\label{ex:bblite} Consider the hybrid system model for the bouncing ball with 
data given in \IfSAE{Example~1.1 in the instructions file}{Example~\ref{ex:bb}}.

For this example, we consider the ball to be bouncing on a floor at zero height. 
The constants for the bouncing ball system are $\gamma = 9.81$ and $\lambda=0.8$.
The following procedure is used to simulate this example in the Lite HyEQ Solver:
\begin{itemize}
\item Inside the MATLAB script {\tt run.m}, initial conditions, simulation horizons, 
      a rule for jumps, ode solver options, and a step size coefficient are defined. 
      The function {\tt HyEQsolver.m} is called in order to run the simulation, 
      and a script for plotting solutions is included.
\item Then the MATLAB functions {\tt f.m, C.m, g.m, D.m} 
      are edited according to the data given above.
\item Finally, the simulation is run by clicking the run button in {\tt run.m} 
      or by calling {\tt run.m} in the MATLAB command window.
\end{itemize}

\begin{figure}[ht]
\begin{center}
\subfigure[Height \label{fig:lite-1}]
{
  \includegraphics[width=0.48\textwidth]{figures/Example_1_2/FlowsAndJumpsHeight}
}
\hfill
\subfigure[Velocity \label{fig:lite-2}]
{
  \includegraphics[width=0.48\textwidth]{figures/Example_1_2/FlowsAndJumpsVelocity}
}
\end{center}
\caption{Solution of Example \ref{ex:bblite}
  %TODO: Replace figures with labels formatted using LaTeX interpreter.
}
\end{figure}

\begin{figure}[ht]
  \begin{center}
  \psfrag{t}[c]{$t$}
  \psfrag{j}[c]{$j$}
  \psfrag{x1}[c]{$x_1$}
    \includegraphics[width=.8\textwidth]{figures/Example_1_2/plotHybrid.eps}
    \caption{Hybrid arc corresponding to a solution of Example~\ref{ex:bblite}
      %TODO: Replace figures with labels formatted using LaTeX interpreter.
    }
    \label{fig:lite-3}
  \end{center}
\end{figure} 

Example code for each of the MATLAB files {\tt run.m, f.m, C.m, g.m,} 
and {\tt D.m} is given below.\\
%\scriptsize

% Set the location for MATLAB files included via the "\code" command.
\codeLocation{Matlab2tex_1_2}

\code{run\_ex1\_2.m}
\code{f\_ex1\_2.m}
\code{C\_ex1\_2.m}
\code{g\_ex1\_2.m}
\code{D\_ex1\_2.m}

A solution to the bouncing ball system from 
$x(0,0)=[1,0]^\top$ and with 
{\tt TSPAN = [0 10], JSPAN = [0 20], rule = 1}, 
is depicted in Figure~\ref{fig:lite-1} (height) 
and Figure~\ref{fig:lite-2} (velocity).  
Both the projection onto $t$ and $j$ are shown. 
Figure~\ref{fig:lite-3} depicts the corresponding hybrid arc for the position state.

For MATLAB files of this example, 
see \IfSAE{Examples/Example\_\ref{ex:bblite}}{Examples/Example\_1.2}.



\end{example}


%% ADDED DETAILS OF \matlab{}-based CODE
\subsection{Solver Function}

%\ricardo{Perhaps add some text here to briefly describe what the functions implement?}

The solver function {\tt HyEQsolver} solves the hybrid system using 
three different functions as shown below. 
First, the flows are calculated using the built-in ODE solver function {\tt ODE45} in \matlab{}. 
If the solution leaves the flow set {\tt C}, the discrete event is detected using 
the function {\tt zeroevents} as shown in Section \ref{sec:eventsdetection}. When the state jumps, 
the next value of the state is calculated via the jump map {\tt g} using the function {\tt jump} 
as shown in Section \ref{sec:jumpmap}.\\

\codeLocation{Matlab2tex} 
% \code{HyEQsolver\_inst.m}
% % % This file was automatically created from the m-file 
% "m2tex.m" written by USL. 
% The fontencoding in this file is UTF-8. 
%  
% You will need to include the following two packages in 
% your LaTeX-Main-File. 
%  
% \usepackage{color} 
% \usepackage{fancyvrb} 
%  
% It is advised to use the following option for Inputenc 
% \usepackage[utf8]{inputenc} 
%  
  
% definition of matlab colors: 
\definecolor{mblue}{rgb}{0,0,1} 
\definecolor{mgreen}{rgb}{0.13333,0.5451,0.13333} 
\definecolor{mred}{rgb}{0.62745,0.12549,0.94118} 
\definecolor{mgrey}{rgb}{0.5,0.5,0.5} 
\definecolor{mdarkgrey}{rgb}{0.25,0.25,0.25} 
  
\DefineShortVerb[fontfamily=courier,fontseries=m]{\$} 
\DefineShortVerb[fontfamily=courier,fontseries=b]{\#} 
  
\noindent                                                                                                                                                                                                                                                                            
 \hspace*{-1.6em}{\scriptsize 1}$  $\color{mblue}$function$\color{black}$ [t j x] = HyEQsolver(f,g,C,D,x0,TSPAN,JSPAN,rule,options,solver,E)$\\
 \hspace*{-1.6em}{\scriptsize 2}$  $\color{mgreen}$%HYEQSOLVER solves hybrid equations.$\color{black}$$\\
 \hspace*{-1.6em}{\scriptsize 3}$  $\color{mgreen}$%   Syntax: [t j x] = HYEQSOLVER(f,g,C,D,x0,TSPAN,JSPAN,rule,options,solver,E)$\color{black}$$\\
 \hspace*{-1.6em}{\scriptsize 4}$  $\color{mgreen}$%   computes solutions to the hybrid equations$\color{black}$$\\
 \hspace*{-1.6em}{\scriptsize 5}$  $\color{mgreen}$%$\color{black}$$\\
 \hspace*{-1.6em}{\scriptsize 6}$  $\color{mgreen}$%   \dot{x} = f(x,t,j)  x \in C x^+ = g(x,t,j)  x \in D$\color{black}$$\\
 \hspace*{-1.6em}{\scriptsize 7}$  $\color{mgreen}$%$\color{black}$$\\
 \hspace*{-1.6em}{\scriptsize 8}$  $\color{mgreen}$%   where x is the state, f is the flow map, g is the jump map, C is the$\color{black}$$\\
 \hspace*{-1.6em}{\scriptsize 9}$  $\color{mgreen}$%   flow set, and D is the jump set. It outputs the state trajectory (t,j)$\color{black}$$\\
 \hspace*{-2em}{\scriptsize 10}$  $\color{mgreen}$%   -> x(t,j), where t is the flow time parameter and j is the jump$\color{black}$$\\
 \hspace*{-2em}{\scriptsize 11}$  $\color{mgreen}$%   parameter.$\color{black}$$\\
 \hspace*{-2em}{\scriptsize 12}$  $\color{mgreen}$%$\color{black}$$\\
 \hspace*{-2em}{\scriptsize 13}$  $\color{mgreen}$%   x0 defines the initial condition for the state.$\color{black}$$\\
 \hspace*{-2em}{\scriptsize 14}$  $\color{mgreen}$%$\color{black}$$\\
 \hspace*{-2em}{\scriptsize 15}$  $\color{mgreen}$%   TSPAN = [TSTART TFINAL] is the time interval. JSPAN = [JSTART JSTOP] is$\color{black}$$\\
 \hspace*{-2em}{\scriptsize 16}$  $\color{mgreen}$%       the interval for discrete jumps. The algorithm stop when the first$\color{black}$$\\
 \hspace*{-2em}{\scriptsize 17}$  $\color{mgreen}$%       stop condition is reached.$\color{black}$$\\
 \hspace*{-2em}{\scriptsize 18}$  $\color{mgreen}$%$\color{black}$$\\
 \hspace*{-2em}{\scriptsize 19}$  $\color{mgreen}$%   rule (optional parameter) - rule for jumps$\color{black}$$\\
 \hspace*{-2em}{\scriptsize 20}$  $\color{mgreen}$%       rule = 1 (default) -> priority for jumps rule = 2 -> priority for$\color{black}$$\\
 \hspace*{-2em}{\scriptsize 21}$  $\color{mgreen}$%       flows$\color{black}$$\\
 \hspace*{-2em}{\scriptsize 22}$  $\color{mgreen}$%$\color{black}$$\\
 \hspace*{-2em}{\scriptsize 23}$  $\color{mgreen}$%   options (optional parameter) - options for the solver see odeset f.ex.$\color{black}$$\\
 \hspace*{-2em}{\scriptsize 24}$  $\color{mgreen}$%       options = odeset('RelTol',1e-6);$\color{black}$$\\
 \hspace*{-2em}{\scriptsize 25}$  $\color{mgreen}$%       options = odeset('InitialStep',eps);$\color{black}$$\\
 \hspace*{-2em}{\scriptsize 26}$  $\color{mgreen}$%$\color{black}$$\\
 \hspace*{-2em}{\scriptsize 27}$  $\color{mgreen}$%   solver (optional parameter. String) - selection of the desired ode$\color{black}$$\\
 \hspace*{-2em}{\scriptsize 28}$  $\color{mgreen}$%       solver. All ode solvers are suported, exept for ode15i.  See help$\color{black}$$\\
 \hspace*{-2em}{\scriptsize 29}$  $\color{mgreen}$%       odeset for detailed information.$\color{black}$$\\
 \hspace*{-2em}{\scriptsize 30}$  $\color{mgreen}$%$\color{black}$$\\
 \hspace*{-2em}{\scriptsize 31}$  $\color{mgreen}$%   E (optional parameter) - Mass matrix [constant matrix | function_handle]$\color{black}$$\\
 \hspace*{-2em}{\scriptsize 32}$  $\color{mgreen}$%       For problems: $\color{black}$$\\
 \hspace*{-2em}{\scriptsize 33}$  $\color{mgreen}$%       E*\dot{x} = f(x) x \in C $\color{black}$$\\
 \hspace*{-2em}{\scriptsize 34}$  $\color{mgreen}$%       x^+ = g(x)  x \in D$\color{black}$$\\
 \hspace*{-2em}{\scriptsize 35}$  $\color{mgreen}$%       set this property to the value of the constant mass matrix. For$\color{black}$$\\
 \hspace*{-2em}{\scriptsize 36}$  $\color{mgreen}$%       problems with time- or state-dependent mass matrices, set this$\color{black}$$\\
 \hspace*{-2em}{\scriptsize 37}$  $\color{mgreen}$%       property to a function that evaluates the mass matrix. See help$\color{black}$$\\
 \hspace*{-2em}{\scriptsize 38}$  $\color{mgreen}$%       odeset for detailed information.$\color{black}$$\\
 \hspace*{-2em}{\scriptsize 39}$  $\color{mgreen}$%$\color{black}$$\\
 \hspace*{-2em}{\scriptsize 40}$  $\color{mgreen}$%         Example: Bouncing ball with Lite HyEQ Solver$\color{black}$$\\
 \hspace*{-2em}{\scriptsize 41}$  $\color{mgreen}$%$\color{black}$$\\
 \hspace*{-2em}{\scriptsize 42}$  $\color{mgreen}$%         % Consider the hybrid system model for the bouncing ball with data given in$\color{black}$$\\
 \hspace*{-2em}{\scriptsize 43}$  $\color{mgreen}$%         % Example 1.2. For this example, we consider the ball to be bouncing on a$\color{black}$$\\
 \hspace*{-2em}{\scriptsize 44}$  $\color{mgreen}$%         % floor at zero height. The constants for the bouncing ball system are$\color{black}$$\\
 \hspace*{-2em}{\scriptsize 45}$  $\color{mgreen}$%         % \gamma=9.81 and \lambda=0.8. The following procedure is used to$\color{black}$$\\
 \hspace*{-2em}{\scriptsize 46}$  $\color{mgreen}$%         % simulate this example in the Lite HyEQ Solver:$\color{black}$$\\
 \hspace*{-2em}{\scriptsize 47}$  $\color{mgreen}$%$\color{black}$$\\
 \hspace*{-2em}{\scriptsize 48}$  $\color{mgreen}$%         % * Inside the MATLAB script run_ex1_2.m, initial conditions, simulation$\color{black}$$\\
 \hspace*{-2em}{\scriptsize 49}$  $\color{mgreen}$%         % horizons, a rule for jumps, ode solver options, and a step size$\color{black}$$\\
 \hspace*{-2em}{\scriptsize 50}$  $\color{mgreen}$%         % coefficient are defined. The function HYEQSOLVER.m is called in order to$\color{black}$$\\
 \hspace*{-2em}{\scriptsize 51}$  $\color{mgreen}$%         % run the simulation, and a script for plotting solutions is included.$\color{black}$$\\
 \hspace*{-2em}{\scriptsize 52}$  $\color{mgreen}$%         % * Then the MATLAB functions f_ex1_2.m, C_ex1_2.m, g_ex1_2.m, D_ex1_2.m$\color{black}$$\\
 \hspace*{-2em}{\scriptsize 53}$  $\color{mgreen}$%         % are edited according to the data given below.$\color{black}$$\\
 \hspace*{-2em}{\scriptsize 54}$  $\color{mgreen}$%         % * Finally, the simulation is run by clicking the run button in$\color{black}$$\\
 \hspace*{-2em}{\scriptsize 55}$  $\color{mgreen}$%         % run_ex1_2.m or by calling run_ex1_2.m in the MATLAB command window.$\color{black}$$\\
 \hspace*{-2em}{\scriptsize 56}$  $\color{mgreen}$%$\color{black}$$\\
 \hspace*{-2em}{\scriptsize 57}$  $\color{mgreen}$%         % For further information, type in the command window:$\color{black}$$\\
 \hspace*{-2em}{\scriptsize 58}$  $\color{mgreen}$%         web(['Example_1_2.html']);$\color{black}$$\\
 \hspace*{-2em}{\scriptsize 59}$  $\color{mgreen}$%$\color{black}$$\\
 \hspace*{-2em}{\scriptsize 60}$  $\color{mgreen}$%         % Define initial conditions$\color{black}$$\\
 \hspace*{-2em}{\scriptsize 61}$  $\color{mgreen}$%         x1_0 = 1;$\color{black}$$\\
 \hspace*{-2em}{\scriptsize 62}$  $\color{mgreen}$%         x2_0 = 0;$\color{black}$$\\
 \hspace*{-2em}{\scriptsize 63}$  $\color{mgreen}$%         x0   = [x1_0; x2_0];$\color{black}$$\\
 \hspace*{-2em}{\scriptsize 64}$  $\color{mgreen}$%$\color{black}$$\\
 \hspace*{-2em}{\scriptsize 65}$  $\color{mgreen}$%         % Set simulation horizon$\color{black}$$\\
 \hspace*{-2em}{\scriptsize 66}$  $\color{mgreen}$%         TSPAN = [0 10];$\color{black}$$\\
 \hspace*{-2em}{\scriptsize 67}$  $\color{mgreen}$%         JSPAN = [0 20];$\color{black}$$\\
 \hspace*{-2em}{\scriptsize 68}$  $\color{mgreen}$%$\color{black}$$\\
 \hspace*{-2em}{\scriptsize 69}$  $\color{mgreen}$%         % Set rule for jumps and ODE solver options$\color{black}$$\\
 \hspace*{-2em}{\scriptsize 70}$  $\color{mgreen}$%         %$\color{black}$$\\
 \hspace*{-2em}{\scriptsize 71}$  $\color{mgreen}$%         % rule = 1 -> priority for jumps$\color{black}$$\\
 \hspace*{-2em}{\scriptsize 72}$  $\color{mgreen}$%         %$\color{black}$$\\
 \hspace*{-2em}{\scriptsize 73}$  $\color{mgreen}$%         % rule = 2 -> priority for flows$\color{black}$$\\
 \hspace*{-2em}{\scriptsize 74}$  $\color{mgreen}$%         %$\color{black}$$\\
 \hspace*{-2em}{\scriptsize 75}$  $\color{mgreen}$%         % set the maximum step length. At each run of the$\color{black}$$\\
 \hspace*{-2em}{\scriptsize 76}$  $\color{mgreen}$%         % integrator the option 'MaxStep' is set to$\color{black}$$\\
 \hspace*{-2em}{\scriptsize 77}$  $\color{mgreen}$%         % (time length of last integration)*maxStepCoefficient.$\color{black}$$\\
 \hspace*{-2em}{\scriptsize 78}$  $\color{mgreen}$%         %  Default value = 0.1$\color{black}$$\\
 \hspace*{-2em}{\scriptsize 79}$  $\color{mgreen}$%$\color{black}$$\\
 \hspace*{-2em}{\scriptsize 80}$  $\color{mgreen}$%         rule               = 1;$\color{black}$$\\
 \hspace*{-2em}{\scriptsize 81}$  $\color{mgreen}$%$\color{black}$$\\
 \hspace*{-2em}{\scriptsize 82}$  $\color{mgreen}$%         options            = odeset('RelTol',1e-6,'MaxStep',.1);$\color{black}$$\\
 \hspace*{-2em}{\scriptsize 83}$  $\color{mgreen}$%$\color{black}$$\\
 \hspace*{-2em}{\scriptsize 84}$  $\color{mgreen}$%         % Simulate using the HYEQSOLVER script$\color{black}$$\\
 \hspace*{-2em}{\scriptsize 85}$  $\color{mgreen}$%         % Given the matlab functions that models the flow map, jump map,$\color{black}$$\\
 \hspace*{-2em}{\scriptsize 86}$  $\color{mgreen}$%         % flow set and jump set (f_ex1_2, g_ex1_2, C_ex1_2, and D_ex1_2$\color{black}$$\\
 \hspace*{-2em}{\scriptsize 87}$  $\color{mgreen}$%         % respectively)$\color{black}$$\\
 \hspace*{-2em}{\scriptsize 88}$  $\color{mgreen}$%$\color{black}$$\\
 \hspace*{-2em}{\scriptsize 89}$  $\color{mgreen}$%         [t j x] = HYEQSOLVER( @f_ex1_2,@g_ex1_2,@C_ex1_2,@D_ex1_2,...$\color{black}$$\\
 \hspace*{-2em}{\scriptsize 90}$  $\color{mgreen}$%             x0,TSPAN,JSPAN,rule,options,'ode45');$\color{black}$$\\
 \hspace*{-2em}{\scriptsize 91}$  $\color{mgreen}$%$\color{black}$$\\
 \hspace*{-2em}{\scriptsize 92}$  $\color{mgreen}$%         % plot solution$\color{black}$$\\
 \hspace*{-2em}{\scriptsize 93}$  $\color{mgreen}$%$\color{black}$$\\
 \hspace*{-2em}{\scriptsize 94}$  $\color{mgreen}$%         figure(1) % position$\color{black}$$\\
 \hspace*{-2em}{\scriptsize 95}$  $\color{mgreen}$%         clf$\color{black}$$\\
 \hspace*{-2em}{\scriptsize 96}$  $\color{mgreen}$%         subplot(2,1,1),plotflows(t,j,x(:,1))$\color{black}$$\\
 \hspace*{-2em}{\scriptsize 97}$  $\color{mgreen}$%         grid on$\color{black}$$\\
 \hspace*{-2em}{\scriptsize 98}$  $\color{mgreen}$%         ylabel('x1')$\color{black}$$\\
 \hspace*{-2em}{\scriptsize 99}$  $\color{mgreen}$%$\color{black}$$\\
 \hspace*{-2.4em}{\scriptsize 100}$  $\color{mgreen}$%         subplot(2,1,2),plotjumps(t,j,x(:,1))$\color{black}$$\\
 \hspace*{-2.4em}{\scriptsize 101}$  $\color{mgreen}$%         grid on$\color{black}$$\\
 \hspace*{-2.4em}{\scriptsize 102}$  $\color{mgreen}$%         ylabel('x1')$\color{black}$$\\
 \hspace*{-2.4em}{\scriptsize 103}$  $\color{mgreen}$%$\color{black}$$\\
 \hspace*{-2.4em}{\scriptsize 104}$  $\color{mgreen}$%         figure(2) % velocity$\color{black}$$\\
 \hspace*{-2.4em}{\scriptsize 105}$  $\color{mgreen}$%         clf$\color{black}$$\\
 \hspace*{-2.4em}{\scriptsize 106}$  $\color{mgreen}$%         subplot(2,1,1),plotflows(t,j,x(:,2))$\color{black}$$\\
 \hspace*{-2.4em}{\scriptsize 107}$  $\color{mgreen}$%         grid on$\color{black}$$\\
 \hspace*{-2.4em}{\scriptsize 108}$  $\color{mgreen}$%         ylabel('x2')$\color{black}$$\\
 \hspace*{-2.4em}{\scriptsize 109}$  $\color{mgreen}$%$\color{black}$$\\
 \hspace*{-2.4em}{\scriptsize 110}$  $\color{mgreen}$%         subplot(2,1,2),plotjumps(t,j,x(:,2))$\color{black}$$\\
 \hspace*{-2.4em}{\scriptsize 111}$  $\color{mgreen}$%         grid on$\color{black}$$\\
 \hspace*{-2.4em}{\scriptsize 112}$  $\color{mgreen}$%         ylabel('x2')$\color{black}$$\\
 \hspace*{-2.4em}{\scriptsize 113}$  $\color{mgreen}$%$\color{black}$$\\
 \hspace*{-2.4em}{\scriptsize 114}$  $\color{mgreen}$%         % plot hybrid arc$\color{black}$$\\
 \hspace*{-2.4em}{\scriptsize 115}$  $\color{mgreen}$%         $\color{black}$$\\
 \hspace*{-2.4em}{\scriptsize 116}$  $\color{mgreen}$%         figure(3)$\color{black}$$\\
 \hspace*{-2.4em}{\scriptsize 117}$  $\color{mgreen}$%         plotHybridArc(t,j,x)$\color{black}$$\\
 \hspace*{-2.4em}{\scriptsize 118}$  $\color{mgreen}$%         xlabel('j')$\color{black}$$\\
 \hspace*{-2.4em}{\scriptsize 119}$  $\color{mgreen}$%         ylabel('t')$\color{black}$$\\
 \hspace*{-2.4em}{\scriptsize 120}$  $\color{mgreen}$%         zlabel('x1')$\color{black}$$\\
 \hspace*{-2.4em}{\scriptsize 121}$  $\color{mgreen}$%$\color{black}$$\\
 \hspace*{-2.4em}{\scriptsize 122}$  $\color{mgreen}$%         % plot solution using plotHarc and plotHarcColor$\color{black}$$\\
 \hspace*{-2.4em}{\scriptsize 123}$  $\color{mgreen}$%$\color{black}$$\\
 \hspace*{-2.4em}{\scriptsize 124}$  $\color{mgreen}$%         figure(4) % position$\color{black}$$\\
 \hspace*{-2.4em}{\scriptsize 125}$  $\color{mgreen}$%         clf$\color{black}$$\\
 \hspace*{-2.4em}{\scriptsize 126}$  $\color{mgreen}$%         subplot(2,1,1), plotHarc(t,j,x(:,1));$\color{black}$$\\
 \hspace*{-2.4em}{\scriptsize 127}$  $\color{mgreen}$%         grid on$\color{black}$$\\
 \hspace*{-2.4em}{\scriptsize 128}$  $\color{mgreen}$%         ylabel('x_1 position')$\color{black}$$\\
 \hspace*{-2.4em}{\scriptsize 129}$  $\color{mgreen}$%         subplot(2,1,2), plotHarc(t,j,x(:,2));$\color{black}$$\\
 \hspace*{-2.4em}{\scriptsize 130}$  $\color{mgreen}$%         grid on$\color{black}$$\\
 \hspace*{-2.4em}{\scriptsize 131}$  $\color{mgreen}$%         ylabel('x_2 velocity')$\color{black}$$\\
 \hspace*{-2.4em}{\scriptsize 132}$  $\color{mgreen}$%$\color{black}$$\\
 \hspace*{-2.4em}{\scriptsize 133}$  $\color{mgreen}$%$\color{black}$$\\
 \hspace*{-2.4em}{\scriptsize 134}$  $\color{mgreen}$%         % plot a phase plane$\color{black}$$\\
 \hspace*{-2.4em}{\scriptsize 135}$  $\color{mgreen}$%         figure(5) % position$\color{black}$$\\
 \hspace*{-2.4em}{\scriptsize 136}$  $\color{mgreen}$%         clf$\color{black}$$\\
 \hspace*{-2.4em}{\scriptsize 137}$  $\color{mgreen}$%         plotHarcColor(x(:,1),j,x(:,2),t);$\color{black}$$\\
 \hspace*{-2.4em}{\scriptsize 138}$  $\color{mgreen}$%         xlabel('x_1')$\color{black}$$\\
 \hspace*{-2.4em}{\scriptsize 139}$  $\color{mgreen}$%         ylabel('x_2')$\color{black}$$\\
 \hspace*{-2.4em}{\scriptsize 140}$  $\color{mgreen}$%         grid on$\color{black}$$\\
 \hspace*{-2.4em}{\scriptsize 141}$  $\color{mgreen}$%$\color{black}$$\\
 \hspace*{-2.4em}{\scriptsize 142}$  $\color{mgreen}$%--------------------------------------------------------------------------$\color{black}$$\\
 \hspace*{-2.4em}{\scriptsize 143}$  $\color{mgreen}$% Matlab M-file Project: HyEQ Toolbox @  Hybrid Systems Laboratory (HSL),$\color{black}$$\\
 \hspace*{-2.4em}{\scriptsize 144}$  $\color{mgreen}$% https://hybrid.soe.ucsc.edu/software$\color{black}$$\\
 \hspace*{-2.4em}{\scriptsize 145}$  $\color{mgreen}$% http://hybridsimulator.wordpress.com/$\color{black}$$\\
 \hspace*{-2.4em}{\scriptsize 146}$  $\color{mgreen}$% Filename: HYEQSOLVER.m$\color{black}$$\\
 \hspace*{-2.4em}{\scriptsize 147}$  $\color{mgreen}$%--------------------------------------------------------------------------$\color{black}$$\\
 \hspace*{-2.4em}{\scriptsize 148}$  $\color{mgreen}$%   See also HYEQSOLVER, PLOTARC, PLOTARC3, PLOTFLOWS, PLOTHARC,$\color{black}$$\\
 \hspace*{-2.4em}{\scriptsize 149}$  $\color{mgreen}$%   PLOTHARCCOLOR, PLOTHARCCOLOR3D, PLOTHYBRIDARC, PLOTJUMPS.$\color{black}$$\\
 \hspace*{-2.4em}{\scriptsize 150}$  $\color{mgreen}$%   Copyright @ Hybrid Systems Laboratory (HSL),$\color{black}$$\\
 \hspace*{-2.4em}{\scriptsize 151}$  $\color{mgreen}$%   Revision: 0.0.0.4 Date: 04/6/2017 16:26:00$\color{black}$$\\
 \hspace*{-2.4em}{\scriptsize 152}$  $\\
 \hspace*{-2.4em}{\scriptsize 153}$  $\\
 \hspace*{-2.4em}{\scriptsize 154}$  $\color{mblue}$if$\color{black}$ ~exist($\color{mred}$'rule'$\color{black}$,$\color{mred}$'var'$\color{black}$)$\\
 \hspace*{-2.4em}{\scriptsize 155}$      rule = 1;$\\
 \hspace*{-2.4em}{\scriptsize 156}$  $\color{mblue}$end$\color{black}$$\\
 \hspace*{-2.4em}{\scriptsize 157}$  $\\
 \hspace*{-2.4em}{\scriptsize 158}$  $\color{mblue}$if$\color{black}$ ~exist($\color{mred}$'options'$\color{black}$,$\color{mred}$'var'$\color{black}$)$\\
 \hspace*{-2.4em}{\scriptsize 159}$      options = odeset();$\\
 \hspace*{-2.4em}{\scriptsize 160}$  $\color{mblue}$end$\color{black}$$\\
 \hspace*{-2.4em}{\scriptsize 161}$  $\color{mblue}$if$\color{black}$ exist($\color{mred}$'E'$\color{black}$,$\color{mred}$$\color{mred}$'var'$\color{black}$$\color{black}$) && ~exist($\color{mred}$'solver'$\color{black}$,$\color{mred}$$\color{mred}$'var'$\color{black}$$\color{black}$)$\\
 \hspace*{-2.4em}{\scriptsize 162}$      solver = $\color{mred}$'ode15s'$\color{black}$;$\\
 \hspace*{-2.4em}{\scriptsize 163}$  $\color{mblue}$end$\color{black}$$\\
 \hspace*{-2.4em}{\scriptsize 164}$  $\color{mblue}$if$\color{black}$ ~exist($\color{mred}$'solver'$\color{black}$,$\color{mred}$'var'$\color{black}$)$\\
 \hspace*{-2.4em}{\scriptsize 165}$      solver = $\color{mred}$'ode45'$\color{black}$;$\\
 \hspace*{-2.4em}{\scriptsize 166}$  $\color{mblue}$end$\color{black}$$\\
 \hspace*{-2.4em}{\scriptsize 167}$  $\color{mblue}$if$\color{black}$ ~exist($\color{mred}$'E'$\color{black}$,$\color{mred}$'var'$\color{black}$)$\\
 \hspace*{-2.4em}{\scriptsize 168}$      E = [];$\\
 \hspace*{-2.4em}{\scriptsize 169}$  $\color{mblue}$end$\color{black}$$\\
 \hspace*{-2.4em}{\scriptsize 170}$  $\color{mgreen}$% mass matrix (if existent)$\color{black}$$\\
 \hspace*{-2.4em}{\scriptsize 171}$  isDAE = false;$\\
 \hspace*{-2.4em}{\scriptsize 172}$  $\color{mblue}$if$\color{black}$ ~isempty(E)$\\
 \hspace*{-2.4em}{\scriptsize 173}$      isDAE = true;$\\
 \hspace*{-2.4em}{\scriptsize 174}$      $\color{mblue}$switch$\color{black}$ isa(E,$\color{mred}$'function_handle'$\color{black}$)$\\
 \hspace*{-2.4em}{\scriptsize 175}$          $\color{mblue}$case$\color{black}$ true $\color{mgreen}$% Function E(x)$\color{black}$$\\
 \hspace*{-2.4em}{\scriptsize 176}$              M = E;$\\
 \hspace*{-2.4em}{\scriptsize 177}$              options = odeset(options,$\color{mred}$'Mass'$\color{black}$,M,$\color{mred}$'Stats'$\color{black}$,$\color{mred}$'off'$\color{black}$,...$\\
 \hspace*{-2.4em}{\scriptsize 178}$                  $\color{mred}$'MassSingular'$\color{black}$,$\color{mred}$'maybe'$\color{black}$,$\color{mred}$'MStateDependence'$\color{black}$,$\color{mred}$'strong'$\color{black}$,...$\\
 \hspace*{-2.4em}{\scriptsize 179}$                  $\color{mred}$'InitialSlope'$\color{black}$,f_hdae(x0,TSPAN(1))); $\\
 \hspace*{-2.4em}{\scriptsize 180}$          $\color{mblue}$case$\color{black}$ false $\color{mgreen}$% Constant double matrix$\color{black}$$\\
 \hspace*{-2.4em}{\scriptsize 181}$              M = double(E);$\\
 \hspace*{-2.4em}{\scriptsize 182}$              options = odeset(options,$\color{mred}$'Mass'$\color{black}$,M,$\color{mred}$'Stats'$\color{black}$,$\color{mred}$'off'$\color{black}$,...$\\
 \hspace*{-2.4em}{\scriptsize 183}$                  $\color{mred}$'MassSingular'$\color{black}$,$\color{mred}$'maybe'$\color{black}$,$\color{mred}$'MStateDependence'$\color{black}$,$\color{mred}$'none'$\color{black}$);$\\
 \hspace*{-2.4em}{\scriptsize 184}$      $\color{mblue}$end$\color{black}$$\\
 \hspace*{-2.4em}{\scriptsize 185}$  $\color{mblue}$end$\color{black}$$\\
 \hspace*{-2.4em}{\scriptsize 186}$  $\\
 \hspace*{-2.4em}{\scriptsize 187}$  odeX = str2func(solver);$\\
 \hspace*{-2.4em}{\scriptsize 188}$  nargf = nargin(f);$\\
 \hspace*{-2.4em}{\scriptsize 189}$  nargg = nargin(g);$\\
 \hspace*{-2.4em}{\scriptsize 190}$  nargC = nargin(C);$\\
 \hspace*{-2.4em}{\scriptsize 191}$  nargD = nargin(D);$\\
 \hspace*{-2.4em}{\scriptsize 192}$  $\\
 \hspace*{-2.4em}{\scriptsize 193}$  $\\
 \hspace*{-2.4em}{\scriptsize 194}$  $\\
 \hspace*{-2.4em}{\scriptsize 195}$  $\color{mgreen}$% simulation horizon$\color{black}$$\\
 \hspace*{-2.4em}{\scriptsize 196}$  tstart = TSPAN(1);$\\
 \hspace*{-2.4em}{\scriptsize 197}$  tfinal = TSPAN(end);$\\
 \hspace*{-2.4em}{\scriptsize 198}$  jout = JSPAN(1);$\\
 \hspace*{-2.4em}{\scriptsize 199}$  j = jout(end);$\\
 \hspace*{-2.4em}{\scriptsize 200}$  $\\
 \hspace*{-2.4em}{\scriptsize 201}$  $\color{mgreen}$% simulate$\color{black}$$\\
 \hspace*{-2.4em}{\scriptsize 202}$  tout = tstart;$\\
 \hspace*{-2.4em}{\scriptsize 203}$  [rx,cx] = size(x0);$\\
 \hspace*{-2.4em}{\scriptsize 204}$  $\color{mblue}$if$\color{black}$ rx == 1$\\
 \hspace*{-2.4em}{\scriptsize 205}$      xout = x0;$\\
 \hspace*{-2.4em}{\scriptsize 206}$  $\color{mblue}$elseif$\color{black}$ cx == 1$\\
 \hspace*{-2.4em}{\scriptsize 207}$      xout = x0.';$\\
 \hspace*{-2.4em}{\scriptsize 208}$  $\color{mblue}$else$\color{black}$$\\
 \hspace*{-2.4em}{\scriptsize 209}$      error($\color{mred}$'Error, x0 does not have the proper size'$\color{black}$)$\\
 \hspace*{-2.4em}{\scriptsize 210}$  $\color{mblue}$end$\color{black}$$\\
 \hspace*{-2.4em}{\scriptsize 211}$  $\\
 \hspace*{-2.4em}{\scriptsize 212}$  $\color{mgreen}$% Jump if jump is prioritized:$\color{black}$$\\
 \hspace*{-2.4em}{\scriptsize 213}$  $\color{mblue}$if$\color{black}$ rule == 1$\\
 \hspace*{-2.4em}{\scriptsize 214}$      $\color{mblue}$while$\color{black}$ (j<JSPAN(end))$\\
 \hspace*{-2.4em}{\scriptsize 215}$          $\color{mgreen}$% Check if value it is possible to jump current position$\color{black}$$\\
 \hspace*{-2.4em}{\scriptsize 216}$          insideD = fun_wrap(xout(end,:).',tout(end),j,D,nargD);$\\
 \hspace*{-2.4em}{\scriptsize 217}$          $\color{mblue}$if$\color{black}$ insideD == 1$\\
 \hspace*{-2.4em}{\scriptsize 218}$              [j $\color{mred}$tout jout xout] = jump(g,j,tout,jout,xout,nargg);$\color{black}$$\\
 \hspace*{-2.4em}{\scriptsize 219}$          $\color{mblue}$else$\color{black}$$\\
 \hspace*{-2.4em}{\scriptsize 220}$              break;$\\
 \hspace*{-2.4em}{\scriptsize 221}$          $\color{mblue}$end$\color{black}$$\\
 \hspace*{-2.4em}{\scriptsize 222}$      $\color{mblue}$end$\color{black}$$\\
 \hspace*{-2.4em}{\scriptsize 223}$  $\color{mblue}$end$\color{black}$$\\
 \hspace*{-2.4em}{\scriptsize 224}$  fprintf($\color{mred}$'Completed: %3.0f%%'$\color{black}$,0);$\\
 \hspace*{-2.4em}{\scriptsize 225}$  $\color{mblue}$while$\color{black}$ (j < JSPAN(end) && tout(end) < TSPAN(end))$\\
 \hspace*{-2.4em}{\scriptsize 226}$      options = odeset(options,$\color{mred}$'Events'$\color{black}$,@(t,x) zeroevents(x,t,j,C,D,...$\\
 \hspace*{-2.4em}{\scriptsize 227}$          rule,nargC,nargD));$\\
 \hspace*{-2.4em}{\scriptsize 228}$      $\color{mgreen}$% Check if it is possible to flow from current position$\color{black}$$\\
 \hspace*{-2.4em}{\scriptsize 229}$      insideC = fun_wrap(xout(end,:).',tout(end),j,C,nargC);$\\
 \hspace*{-2.4em}{\scriptsize 230}$      $\color{mblue}$if$\color{black}$ insideC == 1$\\
 \hspace*{-2.4em}{\scriptsize 231}$          $\color{mblue}$if$\color{black}$ isDAE$\\
 \hspace*{-2.4em}{\scriptsize 232}$              options = odeset(options,$\color{mred}$'InitialSlope'$\color{black}$,f(xout(end,:).',tout(end)));$\\
 \hspace*{-2.4em}{\scriptsize 233}$          $\color{mblue}$end$\color{black}$$\\
 \hspace*{-2.4em}{\scriptsize 234}$          [t,x] = odeX(@(t,x) fun_wrap(x,t,j,f,nargf),[tout(end) tfinal],...$\\
 \hspace*{-2.4em}{\scriptsize 235}$              xout(end,:).', $\color{mred}$options);$\color{black}$$\\
 \hspace*{-2.4em}{\scriptsize 236}$          nt = length(t);$\\
 \hspace*{-2.4em}{\scriptsize 237}$          tout = [tout; t];$\\
 \hspace*{-2.4em}{\scriptsize 238}$          xout = [xout; x];$\\
 \hspace*{-2.4em}{\scriptsize 239}$          jout = [jout; j*ones(1,nt)'];$\\
 \hspace*{-2.4em}{\scriptsize 240}$      $\color{mblue}$end$\color{black}$$\\
 \hspace*{-2.4em}{\scriptsize 241}$      $\\
 \hspace*{-2.4em}{\scriptsize 242}$      $\color{mgreen}$%Check if it is possible to jump$\color{black}$$\\
 \hspace*{-2.4em}{\scriptsize 243}$      insideD = fun_wrap(xout(end,:).',tout(end),j,D,nargD);$\\
 \hspace*{-2.4em}{\scriptsize 244}$      $\color{mblue}$if$\color{black}$ insideD == 0$\\
 \hspace*{-2.4em}{\scriptsize 245}$          break;$\\
 \hspace*{-2.4em}{\scriptsize 246}$      $\color{mblue}$else$\color{black}$$\\
 \hspace*{-2.4em}{\scriptsize 247}$          $\color{mblue}$if$\color{black}$ rule == 1$\\
 \hspace*{-2.4em}{\scriptsize 248}$              $\color{mblue}$while$\color{black}$ (j<JSPAN(end))$\\
 \hspace*{-2.4em}{\scriptsize 249}$                  $\color{mgreen}$% Check if it is possible to jump from current position$\color{black}$$\\
 \hspace*{-2.4em}{\scriptsize 250}$                  insideD = fun_wrap(xout(end,:).',tout(end),j,D,nargD);$\\
 \hspace*{-2.4em}{\scriptsize 251}$                  $\color{mblue}$if$\color{black}$ insideD == 1$\\
 \hspace*{-2.4em}{\scriptsize 252}$                      [j $\color{mred}$tout jout xout] = jump(g,j,tout,jout,xout,nargg);$\color{black}$$\\
 \hspace*{-2.4em}{\scriptsize 253}$                  $\color{mblue}$else$\color{black}$$\\
 \hspace*{-2.4em}{\scriptsize 254}$                      break;$\\
 \hspace*{-2.4em}{\scriptsize 255}$                  $\color{mblue}$end$\color{black}$$\\
 \hspace*{-2.4em}{\scriptsize 256}$              $\color{mblue}$end$\color{black}$$\\
 \hspace*{-2.4em}{\scriptsize 257}$          $\color{mblue}$else$\color{black}$$\\
 \hspace*{-2.4em}{\scriptsize 258}$              [j $\color{mred}$tout jout xout] = jump(g,j,tout,jout,xout,nargg);$\color{black}$$\\
 \hspace*{-2.4em}{\scriptsize 259}$          $\color{mblue}$end$\color{black}$$\\
 \hspace*{-2.4em}{\scriptsize 260}$      $\color{mblue}$end$\color{black}$$\\
 \hspace*{-2.4em}{\scriptsize 261}$      fprintf($\color{mred}$'\b\b\b\b%3.0f%%'$\color{black}$,max(100*j/JSPAN(end),100*tout(end)/TSPAN(end)));$\\
 \hspace*{-2.4em}{\scriptsize 262}$  $\color{mblue}$end$\color{black}$$\\
 \hspace*{-2.4em}{\scriptsize 263}$  t = tout;$\\
 \hspace*{-2.4em}{\scriptsize 264}$  x = xout;$\\
 \hspace*{-2.4em}{\scriptsize 265}$  j = jout;$\\
 \hspace*{-2.4em}{\scriptsize 266}$  fprintf($\color{mred}$'\nDone\n'$\color{black}$);$\\
 \hspace*{-2.4em}{\scriptsize 267}$  $\color{mblue}$end$\color{black}$$\\
 \hspace*{-2.4em}{\scriptsize 268}$  $\\ 
  
\UndefineShortVerb{\$} 
\UndefineShortVerb{\#}
% \label{scr:HyEQsolver}

% \subsubsection{Events Detection}
% \label{sec:eventsdetection}

% \code{zeroevents\_inst.m}
% % % This file was automatically created from the m-file 
% "m2tex.m" written by USL. 
% The fontencoding in this file is UTF-8. 
%  
% You will need to include the following two packages in 
% your LaTeX-Main-File. 
%  
% \usepackage{color} 
% \usepackage{fancyvrb} 
%  
% It is advised to use the following option for Inputenc 
% \usepackage[utf8]{inputenc} 
%  
  
% definition of matlab colors: 
\definecolor{mblue}{rgb}{0,0,1} 
\definecolor{mgreen}{rgb}{0.13333,0.5451,0.13333} 
\definecolor{mred}{rgb}{0.62745,0.12549,0.94118} 
\definecolor{mgrey}{rgb}{0.5,0.5,0.5} 
\definecolor{mdarkgrey}{rgb}{0.25,0.25,0.25} 
  
\DefineShortVerb[fontfamily=courier,fontseries=m]{\$} 
\DefineShortVerb[fontfamily=courier,fontseries=b]{\#} 
  
\begin{Verbatim}[commandchars=\$\{\},numbers=left,numbersep=2pt] 

    $textcolor{mblue}{function} [value,isterminal,direction] = zeroevents(x,t,j,C,D,rule,nargC,nargD) 
    $textcolor{mblue}{switch} rule 
        $textcolor{mblue}{case} 1 $textcolor{mgreen}{% -> priority for jumps} 
            isterminal(1) = 1; $textcolor{mgreen}{% InsideC} 
            isterminal(2) = 1; $textcolor{mgreen}{% Inside(C \cap D)} 
            isterminal(3) = 1; $textcolor{mgreen}{% OutsideC} 
            direction(1) = -1; $textcolor{mgreen}{% InsideC} 
            direction(2) = -1; $textcolor{mgreen}{% Inside(C \cap D)} 
            direction(3) =  1; $textcolor{mgreen}{% OutsideC} 
        $textcolor{mblue}{case} 2 $textcolor{mgreen}{%(default) -> priority for flows} 
            isterminal(1) = 1; $textcolor{mgreen}{% InsideC} 
            isterminal(2) = 0; $textcolor{mgreen}{% Inside(C \cap D)} 
            isterminal(3) = 1; $textcolor{mgreen}{% OutsideC} 
            direction(1) = -1; $textcolor{mgreen}{% InsideC} 
            direction(2) = -1; $textcolor{mgreen}{% Inside(C \cap D)} 
            direction(3) =  1; $textcolor{mgreen}{% OutsideC} 
    $textcolor{mblue}{end} 
     
    insideC = fun_wrap(x,t,j,C,nargC); 
    insideD = fun_wrap(x,t,j,D,nargD); 
    outsideC = -fun_wrap(x,t,j,C,nargC); 
     
     
    value(1) = 2*insideC; 
    value(2) = 2-insideC - insideD; 
    value(3) = 2*outsideC; 
     
    $textcolor{mblue}{end} 
      
\end{Verbatim}  
  
\UndefineShortVerb{\$} 
\UndefineShortVerb{\#} 
 
% \label{scr:zeroevents}

% \subsubsection{Jump Map}
% \label{sec:jumpmap}

% \code{jump\_inst.m}
% % % This file was automatically created from the m-file 
% "m2tex.m" written by USL. 
% The fontencoding in this file is UTF-8. 
%  
% You will need to include the following two packages in 
% your LaTeX-Main-File. 
%  
% \usepackage{color} 
% \usepackage{fancyvrb} 
%  
% It is advised to use the following option for Inputenc 
% \usepackage[utf8]{inputenc} 
%  
  
% definition of matlab colors: 
\definecolor{mblue}{rgb}{0,0,1} 
\definecolor{mgreen}{rgb}{0.13333,0.5451,0.13333} 
\definecolor{mred}{rgb}{0.62745,0.12549,0.94118} 
\definecolor{mgrey}{rgb}{0.5,0.5,0.5} 
\definecolor{mdarkgrey}{rgb}{0.25,0.25,0.25} 
  
\DefineShortVerb[fontfamily=courier,fontseries=m]{\$} 
\DefineShortVerb[fontfamily=courier,fontseries=b]{\#} 
  
\noindent          
 \hspace*{-1.6em}{\scriptsize 1}$  $\color{mblue}$function$\color{black}$ [j tout jout xout] = jump(g,j,tout,jout,xout,nargfun)$\\
 \hspace*{-1.6em}{\scriptsize 2}$  $\color{mgreen}$% Jump$\color{black}$$\\
 \hspace*{-1.6em}{\scriptsize 3}$  j = j+1;$\\
 \hspace*{-1.6em}{\scriptsize 4}$  y = fun_wrap(xout(end,:).',tout(end),jout(end),g,nargfun); $\\
 \hspace*{-1.6em}{\scriptsize 5}$  $\color{mgreen}$% Save results$\color{black}$$\\
 \hspace*{-1.6em}{\scriptsize 6}$  tout = [tout; tout(end)];$\\
 \hspace*{-1.6em}{\scriptsize 7}$  xout = [xout; y.'];$\\
 \hspace*{-1.6em}{\scriptsize 8}$  jout = [jout; j];$\\
 \hspace*{-1.6em}{\scriptsize 9}$  $\color{mblue}$end$\color{black}$$\\
 \hspace*{-2em}{\scriptsize 10}$  $\\ 
  
\UndefineShortVerb{\$} 
\UndefineShortVerb{\#}
% \label{scr:jump}

% \subsubsection{Function Wrapper}
% \label{sec:funwrapp}

% \code{fun\_wrap\_inst.m}
% % % This file was automatically created from the m-file 
% "m2tex.m" written by USL. 
% The fontencoding in this file is UTF-8. 
%  
% You will need to include the following two packages in 
% your LaTeX-Main-File. 
%  
% \usepackage{color} 
% \usepackage{fancyvrb} 
%  
% It is advised to use the following option for Inputenc 
% \usepackage[utf8]{inputenc} 
%  
  
% definition of matlab colors: 
\definecolor{mblue}{rgb}{0,0,1} 
\definecolor{mgreen}{rgb}{0.13333,0.5451,0.13333} 
\definecolor{mred}{rgb}{0.62745,0.12549,0.94118} 
\definecolor{mgrey}{rgb}{0.5,0.5,0.5} 
\definecolor{mdarkgrey}{rgb}{0.25,0.25,0.25} 
  
\DefineShortVerb[fontfamily=courier,fontseries=m]{\$} 
\DefineShortVerb[fontfamily=courier,fontseries=b]{\#} 
  
\begin{Verbatim}[commandchars=\$\{\},numbers=left,numbersep=2pt] 

    $textcolor{mblue}{function} xdelta = fun_wrap(x,t,j,h,nargfun) 
    $textcolor{mgreen}{%fun_wrap   Variable input arguments function (easy use for users).} 
    $textcolor{mgreen}{%   fun_wrap(x,t,j,h,nargfun) depending on the function h written by the} 
    $textcolor{mgreen}{%   user, this script selects how the HyEQ solver should call that} 
    $textcolor{mgreen}{%   function.} 
    $textcolor{mgreen}{%    x: state} 
    $textcolor{mgreen}{%    t: time} 
    $textcolor{mgreen}{%    j: discrete time} 
    $textcolor{mgreen}{%    h: function handle} 
    $textcolor{mgreen}{%    nargfun: number of input arguments of function h    } 
    $textcolor{mgreen}{%--------------------------------------------------------------------------} 
    $textcolor{mgreen}{% Matlab M-file Project: HyEQ Toolbox @  Hybrid Systems Laboratory (HSL), } 
    $textcolor{mgreen}{% https://hybrid.soe.ucsc.edu/software} 
    $textcolor{mgreen}{% http://hybridsimulator.wordpress.com/} 
    $textcolor{mgreen}{% Filename: fun_wrap.m} 
    $textcolor{mgreen}{%--------------------------------------------------------------------------} 
    $textcolor{mgreen}{%   See also HYEQSOLVER, PLOTARC, PLOTARC3, PLOTFLOWS, PLOTHARC,} 
    $textcolor{mgreen}{%   PLOTHARCCOLOR, PLOTHARCCOLOR3D, PLOTHYBRIDARC, PLOTJUMPS.} 
    $textcolor{mgreen}{%   Copyright @ Hybrid Systems Laboratory (HSL),} 
    $textcolor{mgreen}{%   Revision: 0.0.0.3 Date: 01/28/2016 5:12:00} 
     
     
    $textcolor{mblue}{switch} nargfun 
        $textcolor{mblue}{case} 1 
            xdelta = h(x); 
        $textcolor{mblue}{case} 2 
            xdelta = h(x,t); 
        $textcolor{mblue}{case} 3 
            xdelta = h(x,t,j);         
    $textcolor{mblue}{end} 
    $textcolor{mblue}{end}  
\end{Verbatim}  
  
\UndefineShortVerb{\$} 
\UndefineShortVerb{\#} 
 
% \label{scr:funwrapp}


\section{Acknowledgments}
\label{sec:acknowledgments}

We would like to thank Giampiero Campa for his thoughtful feedback 
and advice as well as Torstein Ingebrigtsen Bo for his comments 
and initial version of the \matlab{}-based simulator code, 
and the following list of people who have helped to test this toolbox:

\begin{itemize}
\item Cenk Oguz Saglam - University of California, Santa Barbara
\item Bharani Malladi - The University of Arizona
\end{itemize}

\bibliographystyle{unsrt} 
\bibliography{biblio}\label{sec:refs}

\include{foot}

\end{document}
