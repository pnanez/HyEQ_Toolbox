%%%%%%%%%%%%%%%%%%%%%%%%%%%%%%%%%%%%%%%%%%%%%%%%%%%%%%%%%%%%%%%%%%%%%%%%%
%%%%%%%%%%%%%%%%%%%%%%%%%%%%%%%%%%%%%%%%%%%%%%%%%%%%%%%%%%%%%%%%%%%%%%%%%
%%%%%%%%%%%%%%%%%%%%%%%%%%%%%%%%%%%%%%%%%%%%%%%%%%%%%%%%%%%%%%%%%%%%%%%%%
%%%%%  THEOREMS AND ENVIRONMENTS
%%%%%%%%%%%%%%%%%%%%%%%%%%%%%%%%%%%%%%%%%%%%%%%%%%%%%%%%%%%%%%%%%%%%%%%%%

\newtheorem{nntheorem}{ Theorem}[section]
\newtheorem{nnlemma}[nntheorem]{ Lemma}
\newtheorem{nndefinition}[nntheorem]{ Definition}
\newtheorem{nncorollary}[nntheorem]{ Corollary}
\newtheorem{nnproposition}[nntheorem]{ Proposition}
\newtheorem{nnassumption}[nntheorem]{ Assumption}
\newtheorem{nexample}{ Example}[section]
\newtheorem{nnremark}[nntheorem]{ Remark}
\newtheorem{nnproblem}[nntheorem]{ Exercise}

\newenvironment{theorem}[1]
{\begin{nntheorem}{\rm\textrm{(#1)}}\sl}
{\end{nntheorem}}

\newenvironment{proposition}[1]
{\begin{nnproposition}{\rm\textrm{(#1)}}\sl}
{\end{nnproposition}}

\newenvironment{propositionnodes}[1]
{\begin{nnproposition}{\rm\textrm{#1}}\sl}
{\end{nnproposition}}

\newenvironment{lemma}[1]
{\begin{nnlemma}{\rm\textrm{(#1)}}\sl}
{\end{nnlemma}}

\newenvironment{corollary}[1]
{\begin{nncorollary}{\rm\textrm{(#1)}}\sl}
{\end{nncorollary}}

\newenvironment{definition}[1]
{\begin{nndefinition}{\rm\textrm{(#1)}}\sl}
{\end{nndefinition}}

\newenvironment{assumption}[1]
{\begin{nnassumption}{\rm\textrm{(#1)}}\sl}
{\end{nnassumption}}

\newenvironment{assumptionnodes}[1]
{\begin{nnassumption}{\rm\textrm{#1}}\sl}
{\end{nnassumption}}


\newenvironment{remark}[1]
{\begin{nnremark}{\rm\textrm{(#1)}}\sl}
{\end{nnremark}}

\newenvironment{remarknodes}[1]
{\begin{nnremark}{\rm\textrm{#1}}\sl}
{\end{nnremark}}


\newenvironment{problem}[1]
{\begin{nnproblem}{\rm\textrm{(#1)}}\sl}
{\end{nnproblem}}


%%%%%%%%%%%%%%%%%%%%%%%%%%%%%%%%%%%%%%%%%%%%%%%%%%%%%%%%%%%%%%%%%%%%%%%%%

\newcommand{\eoe}
           {\hspace*{\fill}{$\vcenter{\hrule height1pt 
                     \hbox{\vrule width1pt height3pt 
            \kern3pt \vrule width1pt} \hrule height1pt}$} }

\newenvironment{example}[1]
{\begin{nexample}{\rm\textrm{(#1)}}\rm}{\eoe\end{nexample}}

%%%%%%%%%%%%%%%%%%%%%%%%%%%%%%%%%%%%%%%%%%%%%%%%%%%%%%%%%%%%%%%%%%%%%%%%%

\newcommand{\eop}
           {\hspace*{\fill}{$\vcenter{\hrule height1pt 
                     \hbox{\vrule width1pt height5pt 
            \kern5pt \vrule width1pt} \hrule height1pt}$} }

%\newenvironment{proof}
%{\par\noindent\textbf{Proof.}}{\eop\smallskip\vskip 3 pt}


%%%%%%%%%%%%%%%%%%%%%%%%%%%%%%%%%%%%%%%%%%%%%%%%%%%%%%%%%%%%%%%%%%%%%%%%%
%%%%%%%%%%%%%%%%%%%%%%%%%%%%%%%%%%%%%%%%%%%%%%%%%%%%%%%%%%%%%%%%%%%%%%%%%
%%%%%%%%%%%%%%%%%%%%%%%%%%%%%%%%%%%%%%%%%%%%%%%%%%%%%%%%%%%%%%%%%%%%%%%%%
%%%%%  PRETTY OBVIOUS NEWCOMMANDS
%%%%%%%%%%%%%%%%%%%%%%%%%%%%%%%%%%%%%%%%%%%%%%%%%%%%%%%%%%%%%%%%%%%%%%%%%
 
\newcommand{\ball}{{\mathbb B}}
\newcommand{\con}{{\mathop{\rm con}\nolimits}}
\newcommand{\clcon}{{\overline{\con}}}
\newcommand{\dom}{\mathop{\rm dom}\nolimits}
\newcommand{\glim}{\mathop{\rm gph\mbox{-}lim}}
\newcommand{\glimsup}{\mathop{\rm gph\mbox{-}lim\,sup}}
\newcommand{\gliminf}{\mathop{\rm gph\mbox{-}lim\,inf}}
\newcommand{\gph}{\mathop{\rm gph}\nolimits}
\renewcommand{\iint}{\mathop{\rm int}\nolimits}
\newcommand{\integers}{{\mathbb Z}}
\newcommand{\iti}{{i\to\infty}}
\newcommand{\kti}{{k\to\infty}}
\newcommand{\KL}{{{\mathcal{K}\mathcal{L}}}}
\newcommand{\KLL}{{{\mathcal{K}\mathcal{L}\mathcal{L}}}}
\newcommand{\naturals}{{\mathbb N}}
\newcommand{\ox}{{\bar{x}}}
\newcommand{\reals}{{\mathbb R}}
\renewcommand{\Re}{{\mathbb R}}
\newcommand{\realsplus}{{\reals_{\geq 0}}}
\newcommand{\realspplus}{{\reals_{>0}}}
\newcommand{\rge}{\mathop{\rm rge}\nolimits}
\DeclareMathOperator*{\argmin}{\mathop{\rm argmin}}   % Jan Hlavacek
\DeclareMathOperator*{\argmax}{\mathop{\rm argmax}}   % Jan Hlavacek
%\newcommand{\rge}{\mathop{\rm rge}}      
%\newcommand{\rge}{{\mathop{\rm rge}\nolimits}}


% \newcommand{\tto}{\;{\lower 1pt \hbox{$\rightarrow$}}\kern -10pt
%            \hbox{\raise 2pt \hbox{$\rightarrow$}}\;}

\newcommand{\tto}{\;{\lower 1pt \hbox{$\rightarrow$}}\kern -12pt
           \hbox{\raise 2pt \hbox{$\rightarrow$}}\;}



%%%%%%%%%%%%%%%%%%%%%%%%%%%%%%%%%%%%%%%%%%%%%%%%%%%%%%%%%%%%%%%%%%%%%%%%%
%%%%%%%%%%%%%%%%%%%%%%%%%%%%%%%%%%%%%%%%%%%%%%%%%%%%%%%%%%%%%%%%%%%%%%%%%
%%%%%%%%%%%%%%%%%%%%%%%%%%%%%%%%%%%%%%%%%%%%%%%%%%%%%%%%%%%%%%%%%%%%%%%%%
%%%%%  NEWCOMMANDS TO ARGUE ABOUT
%%%%%%%%%%%%%%%%%%%%%%%%%%%%%%%%%%%%%%%%%%%%%%%%%%%%%%%%%%%%%%%%%%%%%%%%%

%% compact attractor
\newcommand{\A}{\mathcal{A}}
%% basin of attraction of a compact attractor
\newcommand{\BA}{\mathcal{B}_\A}
%% generic measuement error
%\newcommand{\e}{e}
%% hybrid system
\newcommand{\HS}{\mathcal{H}}
%% hybrid DAE system
\newcommand{\Hdae}{\mathcal{H}_{DAE}}
%% hybrid system with data
\newcommand{\HSdata}{\HS=(O,F,C,G,D)}
%% hybrid system data only
\newcommand{\data}{(O,F,C,G,D)}
%% hybrid system with data (lower case)
\newcommand{\datal}{(O,f,C,g,D)}
%% hybrid system regularized data only
\newcommand{\regdata}{(O,\reg{F},\reg{C},\reg{G},\reg{D})}
%% hybrid system with measurement error
\newcommand{\HSe}{{\HS_e}}
%% hybrid system, regularized
\newcommand{\HSreg}{{\reg{\HS}}}
%% generic hybrid time domain
\newcommand{\htd}{E}
%% indicator of A 
\newcommand{\indi}{\omega}
%% generic compact set
\newcommand{\K}{K}
%% generic KL function
\newcommand{\kl}{\gamma}
%% length of a hybrid time domain
\newcommand{\length}{\mathop{\rm length}\nolimits}
%% generic set valued mapping
\newcommand{\map}{M}
%% state space
\renewcommand{\O}{O}
%% pre basin of attraction 
\newcommand{\preBA}{{\BA^p}}
%% admissible radius of perturbation
\newcommand{\rad}{\rho}
%% reachable set
%\newcommand{\reach}{\mathcal{R}}
%% regularization of C D F G
\newcommand{\reg}[1]{{\widehat{#1}}}
%% saturation function
\newcommand{\sat}{{\rm sat}}
%% sequence, for example \seq{x}{i} produces \{x_i\}_{i=0}^\infty
\newcommand{\seq}[2]{{\{#1_{#2}\}_{#2=1}^\infty}}
%% subsequence, for example \seq{x}{i}{k} produces \{x_{i_k}\}_{i=0}^\infty
\newcommand{\subseq}[3]{{\{#1_{#2_{#3}}\}_{#3=1}^\infty}}
%% generic set
\newcommand{\set}{S}
%% solution to a hybrid system or just a hybrid arc
\newcommand{\sol}{\phi}
%% solution to a hybrid closed-loop system (control)
\newcommand{\solcl}{\zeta}
%%
\newcommand{\solcldot}{\dot{\zeta}}
%%
\newcommand{\solclplus}{\zeta^+}
%% initial point for solution to a hybrid system 
\newcommand{\solinit}{\xi}
%% set of maximal solutions to a hybrid system 
\newcommand{\So}{{\mathcal{S}}}
%% set of maximal solutions to a hybrid system \HS
\newcommand{\Sol}{{\mathcal{S}_\HS}}
%% continuous time arc
\newcommand{\solc}{z}
%%
\newcommand{\solcdot}{{\dot{\solc}}}
%%
\newcommand{\soldot}{{\dot{\sol}}}
%% discrete time arc
\newcommand{\sold}{z}
%%
\newcommand{\soldplus}{\sold^+}
%%
\newcommand{\solplus}{{\sol^+}}
%% symbol for switching system
\renewcommand{\SS}{\Sigma}
%\renewcommand{\SS}{\mathcal{S}}
%% supremum in time of a hybrid time domain
\newcommand{\supt}{{\sup\nolimits_t}} 
%% supremum in jumps of a hybrid time domain
\newcommand{\supj}{{\sup\nolimits_j}}
%% generic open set 
\newcommand{\U}{\mathcal{U}}
%% text referring to the type of document is being compiled
\newcommand{\book}{thesis}
% set
% \newcommand{\bigbrace}[1]{\left\{#1\right\}}
% \newcommand{\set}[2]{\bigbrace{#1\ \left| \ #2 \right.}}
% \newcommand{\setsmall}[2]{\{#1\ | \ #2 \}}

%% j-th interval for continuous time t
\newcommand{\intj}{I^j}
%% J-th interval for continuous time t
\newcommand{\intJ}{I^J}
%% 0-th interval for continuous time t
\newcommand{\intzero}{I^0}
%% shorthand for varepsilon
\newcommand{\eps}{\varepsilon}
%% shorthand for \overline
\newcommand{\ol}[1]{\overline{#1}}
%% shorthand for \underline
\newcommand{\ul}[1]{\underline{#1}}
%% matrix 
\newcommand{\matt}[1]{\begin{bmatrix}#1\end{bmatrix}} 
%% set definition
\newcommand{\bigbrace}[1]{\left\{#1\right\}}
\newcommand{\defset}[2]{\bigbrace{#1\ \left| \ #2 \right.}}
% sign function
\newcommand{\sign}{{\mathop{\rm sign}\nolimits}}
% floor function
\newcommand{\floor}{{\mathop{\rm floor}\nolimits}}
%
\newcommand{\HBC}{hybrid basic conditions}
% vertical vector
\newcommand{\vect}[1]{{\left(\begin{matrix}#1\end{matrix}\right)}}
% CT plants data
%\newcommand{\fp}{\widetilde{f}}


%% LOCALIZATION
\newcommand{\floc}{f_{loc}}
\newcommand{\gloc}{g_{loc}}
\newcommand{\Cloc}{C_{loc}}
\newcommand{\Dloc}{D_{loc}}


%%% UNIFORM FORMULAS FOR STUFF
%%%%%%%%%%%%%%%%%%%%%%%%%%%%%%

%% hybrid system with no name and eight parameters:
%% variable, flow set, \in or =, flow map, jump set, \in or =, jump map
\newcommand{\hybridsystem}[7]
{
\left\{
{
\setlength\extrarowheight{.2cm}
\begin{array}{c@{\ }c@{\ }ccc@{\ }c@{\ }c}
#1 &\in&#2 & \ & \dot{#1} & #3 & #4\left(#1\right) \cr
#1 &\in&#5 & \ & {#1}^+   & #6 & #7\left(#1\right)   \cr
\end{array}
}
\right.
}

%% hybrid system with no name and six parameters:
%% variable, flow set, flow map, jump set, jump map
%% inclusions only
\newcommand{\hybridinclusion}[5]
{
\hybridsystem{#1}{#2}{\in}{#3}{#4}{\in}{#5}
}


%% hybrid system with a name and nine parameters:
%% name, variable, flow set, \in or =, flow map, jump set, \in or =, jump map
\newcommand{\hybridsystemwithname}[8]
{
#1: \qquad 
\hybridsystemn{#2}{#3}{#4}{#5}{#6}{#7}{#8}
}


%% hybrid system with a name and seven parameters:
%% name, variable, flow map, flow set, jump map, jump set
\newcommand{\hybridinclusionwithname}[6]
{
#1: \qquad 
\hybridinclusion{#2}{#3}{#4}{#5}{#6}
}

%% hybrid system with logical modes 
%% discrete variable, continuous variable, flow set, \in or =, flow map, jump set, \in or =, jump map (data with no subscripts)
\newcommand{\hybridsystemQ}[8]
{
\left\{
{
\setlength\extrarowheight{.2cm}
\begin{array}{c@{\ }c@{\ }ccc@{\ }c@{\ }c}
#2 & \in & {#3}_{#1} & \ & \dot{#2}    & #4 & {#5}_{#1}\left(#2\right) \cr
#2 & \in & {#6}_{#1} & \ & {(#1,#2)}^+ & #7 & {#8}_{#1}\left(#2\right)  \cr
\end{array}
}
\right.
}


%% data of a hybrid system, first line C, f or F, second line D, g or G. parameters
%% variable, label for C, formula for C, f or F etc, formula for F, label for D, formula for D, g or G etc, formula for G, 
\newcommand{\definehybridsystem}[9]
{
\setlength\extrarowheight{.2cm}
\begin{array}{r@{\ }c@{\ }lcr@{\ }c@{\ }l}
\displaystyle{#2} & = & \displaystyle{#3} & \qquad &
\displaystyle{#4\left(#1\right)} & = & \displaystyle{#5} \\
\displaystyle{#6} & = & \displaystyle{#7} & \qquad &
\displaystyle{#8\left(#1\right)} & = & \displaystyle{#9}
\end{array}
}

%% data of a hybrid system with inputs and outputs, first line C, f or F, second line D, g or G. parameters
%% 
%% 1 variable, 2 input, 3 output, 4 flow set, 5 \in or =, 6 flow map, 7 jump set, 8 jump map, 9 output map


\newcommand{\hybridsystemInputsOutput}[9]
{
\left\{
{
%\setlength\extrarowheight{.2cm}
\begin{array}{l@{\ }c@{\ }lcr@{\ }c@{\ }l}
\displaystyle{#1} & #5 & #6 (#1,{#2}_c) & \qquad & (#1,{#2}_c)\in #4\\
\displaystyle{#1}^+ & #5 & #8 (#1,{#2}_d) & \qquad & (#1,{#2}_d)\in #7\\
\displaystyle{#3}_c & = & {#9}_c (#1) & & \\
\displaystyle{#3}_d & = & {#9}_d (#1) & & 
\end{array}
}
\right.
}

%% data of a hybrid system with inputs and outputs, first line C, f or F, second line D, g or G. parameters
%% 
%% 1 variable, 2 input, 3 output, 4 flow set, 5 \in or =, 6 flow map, 7 jump set, 8 jump map, 9 output map


\newcommand{\hybridsystemInOutSimple}[9]
{
\left\{
{
%\setlength\extrarowheight{.2cm}
\begin{array}{l@{\ }c@{\ }lcr@{\ }c@{\ }l}
\displaystyle{#1} & #5 & #6 (#1,{#2}) & \qquad & (#1,{#2})\in #4\\
\displaystyle{#1}^+ & #5 & #8 (#1,{#2}) & \qquad & (#1,{#2})\in #7\\
\displaystyle{#3} & = & {#9} (#1)
\end{array}
}
\right.
}


%% data of a hybrid DAE system with inputs and outputs, first line C, f or F, second line D, g or G. parameters
%% 
%% 1 state (x), 2 variable(xi),3 variable(chi),4 variable(sigma), 5 flow map 1(xi), 6 flow map 2(chi), 7 input(u), 8 output, 9 flow set, 10 \in or =, 11 flow map, 12 jump map 1(xi), 13 jump map 2(chi), 14 map 3(sigma), 15 jump set, 16 jump map, 17 output map, definition (:)

\newcommand{\hDAEIOshort}[9]
{
    \def\state{#1}%
    \def\tempxi{#2}%
    \def\tempchi{#3}%
    \def\tempsigma{#4}%
    \def\tempfxi{#5}%
    \def\tempfchi{#6}%
    \def\tempu{#7}%
    \def\tempy{#8}%
    \def\tempC{#9}%
    \hDAEIOshortcontinued
}
\newcommand{\hDAEIOshortcontinued}[9]
{
    \def\tempIn{#1}%10
    \def\tempF{#2}%11
    \def\tempgxi{#3}%12
    \def\tempgchi{#4}%13
    \def\tempgsigma{#5}%14
    \def\tempD{#6}%15
    \def\tempG{#7}%16
    \def\temph{#8}%17
    \def\tempdef{#9}%17    
%    \def\tempC{#9}%18
\left\{
{
\begin{array}{@{}r@{}@{}c@{}@{}l@{}@{}c@{}@{}cc@{}}
\begin{bmatrix}
E_{\tempsigma} & 0 & 0 \\ 
0&I&0\\ 
0&0&1
\end{bmatrix}
\begin{bmatrix}
\dot{\tempxi}\\ 
\dot{\tempchi}\\ 
\dot{\tempsigma}
\end{bmatrix}
 & {\tempIn} &
 \quad 
 \begin{bmatrix*}[l]
 {\tempfxi}_{\tempsigma}\tempxi + B_{\tempsigma}{\tempu}_c\\ 
 {\tempfchi}(\state,{\tempu}_c)\\ 
  0 \end{bmatrix*}  
 & \quad & (\state,{\tempu}_c)\in {\tempC}\\
\begin{bmatrix}{\tempxi}^+\\  
{\tempchi}^+\\  {\tempsigma}^+
\end{bmatrix} 
& \tempIn & 
\displaystyle\bigcup_{\tilde{\tempsigma}\in {\tempgsigma}(\state,{\tempu}_d)}
\begin{bmatrix*}[l]
{\tempgxi}(\state,\tilde{\tempsigma},{\tempu}_d)\\ 
{\tempgchi} (\state,{\tempu}_d)\\ 
\tilde{\tempsigma}
\end{bmatrix*}
 & \quad & (\state,{\tempu}_d)\in {\tempD}\\
 \tempy_c & = & \temph_c(\state,{\tempu}_c) & & &\\
 \tempy_d & = & \temph_d(\state,{\tempu}_d) & & &
\end{array}
}
\right.
}

%% data of a hybrid DAE system with inputs and outputs, first line C, f or F, second line D, g or G. parameters
%% 
%% 1 state (x), 2 variable(xi),3 variable(chi),4 variable(sigma), 5 flow map 1(xi), 6 flow map 2(chi), 7 input(u), 8 output, 9 flow set, 10 \in or =, 11 flow map, 12 jump map 1(xi), 13 jump map 2(chi), 14 map 3(sigma), 15 jump set, 16 jump map, 17 output map, definition (:)

\newcommand{\hDAEclshort}[9]
{
    \def\state{#1}%
    \def\tempxi{#2}%
    \def\tempchi{#3}%
    \def\tempsigma{#4}%
    \def\tempfxi{#5}%
    \def\tempfchi{#6}%
    \def\tempu{#7}%
    \def\tempy{#8}%
    \def\tempC{#9}%
    \hDAEclshortcontinued
}
\newcommand{\hDAEclshortcontinued}[9]
{
    \def\tempIn{#1}%10
    \def\tempF{#2}%11
    \def\tempgxi{#3}%12
    \def\tempgchi{#4}%13
    \def\tempgsigma{#5}%14
    \def\tempD{#6}%15
    \def\tempG{#7}%16
    \def\temph{#8}%17
    \def\tempdef{#9}%17    
%    \def\tempC{#9}%18
\left\{
{
\begin{array}{@{}rclrrr}
\begin{bmatrix}
E_{\tempsigma} & 0 & 0 \\ 
0&I&0\\ 
0&0&1
\end{bmatrix}
\begin{bmatrix}
\dot{\tempxi}\\ 
\dot{\tempchi}\\ 
\dot{\tempsigma}
\end{bmatrix}
 & {\tempIn} & 
 \tempF(\state,-\kc(\yflow)+\tildeuflow)\\
% \begin{bmatrix*}[l]
% {\tempfxi}_{\tempsigma}\tempxi + B_{\tempsigma}(-\kc(\yflow)+\tildeuflow)\\ 
% {\tempfchi}(\state,{\tempu}_c)\\ 
%  0 \end{bmatrix*}  
   &&\qquad\qquad(\state,-\kc(\yflow)+\tildeuflow)\in {\tempC}\\
\begin{bmatrix}{\tempxi}^+\\  
{\tempchi}^+\\  {\tempsigma}^+
\end{bmatrix} 
& \tempIn & 
 \tempG(\state,0)
%\displaystyle\bigcup_{\tilde{\tempsigma}\in {\tempgsigma}(\state,{\tempu}_d)}
%\begin{bmatrix*}[l]
%{\tempgxi}(\state,\tilde{\tempsigma},{\tempu}_d)\\ 
%{\tempgchi} (\state,{\tempu}_d)\\ 
%\tilde{\tempsigma}
%\end{bmatrix*}
  \qquad\qquad\quad (\state,0)\in {\tempD}\\
 \tempy_c & = & \temph_c(\state,-\kc(\yflow)+\tildeuflow) & & &\\
 \tempy_d & = & \temph_d(\state,0) & & &
\end{array}
}
\right.
}

%% data of a hybrid DAE system with inputs and outputs, first line C, f or F, second line D, g or G. parameters
%% 
%% 1 state (x), 2 variable(xi),3 variable(chi),4 variable(sigma), 5 flow map 1(xi), 6 flow map 2(chi), 7 input(u), 8 output, 9 flow set, 10 \in or =, 11 flow map, 12 jump map 1(xi), 13 jump map 2(chi), 14 map 3(sigma), 15 jump set, 16 jump map, 17 output map, definition (:)

\newcommand{\hDAEInOutput}[9]
{
    \def\state{#1}%
    \def\tempxi{#2}%
    \def\tempchi{#3}%
    \def\tempsigma{#4}%
    \def\tempfxi{#5}%
    \def\tempfchi{#6}%
    \def\tempu{#7}%
    \def\tempy{#8}%
    \def\tempC{#9}%
    \hDAEInOutputcontinued
}
\newcommand{\hDAEInOutputcontinued}[9]
{
    \def\tempIn{#1}%10
    \def\tempF{#2}%11
    \def\tempgxi{#3}%12
    \def\tempgchi{#4}%13
    \def\tempgsigma{#5}%14
    \def\tempD{#6}%15
    \def\tempG{#7}%16
    \def\temph{#8}%17
    \def\tempdef{#9}%17    
%    \def\tempC{#9}%18
\left\{
{
%\setlength\extrarowheight{.2cm}
\begin{array}{r@{\ }c@{\ }lcr@{\ }c@{\ }l}
\begin{bmatrix}
E_{\tempsigma} & 0 & 0 \\ 
0&I&0\\ 
0&0&1
\end{bmatrix}
\begin{bmatrix}
\dot{\tempxi}\\ 
\dot{\tempchi}\\ 
\dot{\tempsigma}
\end{bmatrix}
 & {\tempIn} & \begin{bmatrix*}[l]
 {\tempfxi}_{\tempsigma}\tempxi + B_{\tempsigma}{\tempu}_c\\ 
 {\tempfchi}(\state,{\tempu}_c)\\ 
  0 \end{bmatrix*} =\tempdef {\tempF}(\state,{\tempu}_c) 
 & \quad & (\state,{\tempu}_c)\in {\tempC}\\
\begin{bmatrix}{\tempxi}^+\\  
{\tempchi}^+\\  {\tempsigma}^+
\end{bmatrix} 
& \tempIn & \displaystyle\bigcup_{\tilde{\tempsigma}\in {\tempgsigma}(\state,{\tempu}_d)}
\begin{bmatrix*}[l]
{\tempgxi}(\state,\tilde{\tempsigma},{\tempu}_d)\\ 
{\tempgchi} (\state,{\tempu}_d)\\ 
\tilde{\tempsigma}
\end{bmatrix*}=\tempdef \tempG(\state,\tempu_d) 
 & \quad & (\state,{\tempu}_d)\in {\tempD}\\
 \tempy_c & = & \temph_c(\state,{\tempu}_c) & & &\\
 \tempy_d & = & \temph_d(\state,{\tempu}_d) & & &
\end{array}
}
\right.
}


%% data of a hybrid DAE system with ZERO-inputs and outputs, first line C, f or F, second line D, g or G. parameters
%% 
%% 1 state (x), 2 variable(xi),3 variable(chi),4 variable(sigma), 5 flow map 1(xi), 6 flow map 2(chi), 7 input(u), 8 output, 9 flow set, 10 \in or =, 11 flow map, 12 jump map 1(xi), 13 jump map 2(chi), 14 map 3(sigma), 15 jump set, 16 jump map, 17 output map, definition (:)

\newcommand{\hDAEZeroInOutput}[9]
{
    \def\state{#1}%
    \def\tempxi{#2}%
    \def\tempchi{#3}%
    \def\tempsigma{#4}%
    \def\tempfxi{#5}%
    \def\tempfchi{#6}%
    \def\tempu{#7}%
    \def\tempy{#8}%
    \def\tempC{#9}%
    \hDAEZeroInOutputcontinued
}
\newcommand{\hDAEZeroInOutputcontinued}[9]
{
    \def\tempIn{#1}%10
    \def\tempF{#2}%11
    \def\tempgxi{#3}%12
    \def\tempgchi{#4}%13
    \def\tempgsigma{#5}%14
    \def\tempD{#6}%15
    \def\tempG{#7}%16
    \def\temph{#8}%17
    \def\tempdef{#9}%17    
%    \def\tempC{#9}%18
\left\{
{
%\setlength\extrarowheight{.2cm}
\begin{array}{r@{\ }c@{\ }lcr@{\ }c@{\ }l}
\begin{bmatrix}
E_{\tempsigma} & 0 & 0 \\ 
0&I&0\\ 
0&0&1
\end{bmatrix}
\begin{bmatrix}
\dot{\tempxi}\\ 
\dot{\tempchi}\\ 
\dot{\tempsigma}
\end{bmatrix}
 & {\tempIn} & \begin{bmatrix*}[l]
 {\tempfxi}_{\tempsigma}\tempxi + B_{\tempsigma}{\tempu}\\ 
 {\tempfchi}(\state,{\tempu})\\ 
  0 \end{bmatrix*} =\tempdef {\tempF}(\state,{\tempu}) 
 & \quad & (\state,{\tempu})\in {\tempC}\\
\begin{bmatrix}{\tempxi}^+\\  
{\tempchi}^+\\  {\tempsigma}^+
\end{bmatrix} 
& \tempIn & \displaystyle\bigcup_{\tilde{\tempsigma}\in {\tempgsigma}(\state,{\tempu})}
\begin{bmatrix*}[l]
{\tempgxi}(\state,\tilde{\tempsigma},{\tempu})\\ 
{\tempgchi} (\state,{\tempu})\\ 
\tilde{\tempsigma}
\end{bmatrix*}=\tempdef \tempG(\state,\tempu) 
 & \quad & (\state,{\tempu})\in {\tempD}\\
 \tempy_c & = & \temph(\state,{\tempu}) & & &\\
 \tempy_d & = & \temph(\state,{\tempu}) & & &
\end{array}
}
\right.
}


%% data of a hybrid DAE system with ZERO-inputs and outputs, first line C, f or F, second line D, g or G. parameters
%% 
%% 1 state (x), 2 variable(xi),3 variable(chi),4 variable(sigma), 5 flow map 1(xi), 6 flow map 2(chi), 7 input(u), 8 output, 9 flow set, 10 \in or =, 11 flow map, 12 jump map 1(xi), 13 jump map 2(chi), 14 map 3(sigma), 15 jump set, 16 jump map, 17 output map, definition (:)

\newcommand{\hDAEZeroInOutputShort}[9]
{
    \def\state{#1}%
    \def\tempxi{#2}%
    \def\tempchi{#3}%
    \def\tempsigma{#4}%
    \def\tempfxi{#5}%
    \def\tempfchi{#6}%
    \def\tempu{#7}%
    \def\tempy{#8}%
    \def\tempC{#9}%
    \hDAEZeroInOutputShortcontinued
}
\newcommand{\hDAEZeroInOutputShortcontinued}[9]
{
    \def\tempIn{#1}%10
    \def\tempF{#2}%11
    \def\tempgxi{#3}%12
    \def\tempgchi{#4}%13
    \def\tempgsigma{#5}%14
    \def\tempD{#6}%15
    \def\tempG{#7}%16
    \def\temph{#8}%17
    \def\tempdef{#9}%17    
%    \def\tempC{#9}%18
\left\{
{
%\setlength\extrarowheight{.2cm}
\begin{array}{@{}r@{}@{}c@{}@{}l@{}@{}l@{}@{}cc@{}}
\begin{bmatrix}
E_{\tempsigma} & 0 & 0 \\ 
0&I&0\\ 
0&0&1
\end{bmatrix}
\begin{bmatrix}
\dot{\tempxi}\\ 
\dot{\tempchi}\\ 
\dot{\tempsigma}
\end{bmatrix}
 & {\tempIn} & 
  \begin{bmatrix*}[c]
 {\tempfxi}_{\tempsigma}\tempxi + B_{\tempsigma}{\tempu}\\ 
 {\tempfchi}(\state,{\tempu})\\ 
  0 \end{bmatrix*}
 & \quad\quad\quad & (\state,{\tempu})\in {\tempC}\\
\begin{bmatrix}{\tempxi}^+\\  
{\tempchi}^+\\  {\tempsigma}^+
\end{bmatrix} 
& \tempIn & \displaystyle\bigcup_{\tilde{\tempsigma}\in {\tempgsigma}(\state,{\tempu})}
\begin{bmatrix*}[c]
{\tempgxi}(\state,\tilde{\tempsigma},{\tempu})\\ 
{\tempgchi} (\state,{\tempu})\\ 
\tilde{\tempsigma}
\end{bmatrix*}
 & \quad\quad\quad & (\state,{\tempu})\in {\tempD}\\
 \tempy_c & = & \temph_c(\state,{\tempu}) & & &\\
 \tempy_d & = & \temph_d(\state,{\tempu}) & & &
\end{array}
}
\right.
}

%% data of a hat hybrid DAE system with inputs and outputs, first line C, f or F, second line D, g or G. parameters
%% 
%% 1 state (x), 2 variable(xi),3 variable(chi),4 variable(sigma), 5 flow map 1(xi), 6 flow map 2(chi), 7 input(u), 8 output, 9 flow set, 10 \in or =, 11 flow map, 12 jump map 1(xi), 13 jump map 2(chi), 14 map 3(sigma), 15 jump set, 16 jump map, 17 output map, definition (:)

\newcommand{\hathDAEInOutput}[9]
{
    \def\state{#1}%
    \def\tempxi{#2}%
    \def\tempchi{#3}%
    \def\tempsigma{#4}%
    \def\tempfxi{#5}%
    \def\tempfchi{#6}%
    \def\tempu{#7}%
    \def\tempy{#8}%
    \def\tempC{#9}%
    \hathDAEInOutputcontinued
}
\newcommand{\hathDAEInOutputcontinued}[9]
{
    \def\tempIn{#1}%10
    \def\tempF{#2}%11
    \def\tempgxi{#3}%12
    \def\tempgchi{#4}%13
    \def\tempgsigma{#5}%14
    \def\tempD{#6}%15
    \def\tempG{#7}%16
    \def\temph{#8}%17
    \def\tempdef{#9}%17    
%    \def\tempC{#9}%18
\left\{
{
%\setlength\extrarowheight{.2cm}
\begin{array}{r@{\ }c@{\ }lcr@{\ }c@{\ }l}
\begin{bmatrix}
E_{\tempsigma} & 0 & 0 \\ 
0&I&0\\ 
0&0&1
\end{bmatrix}
\begin{bmatrix}
\dot{\tempxi}\\ 
\dot{\tempchi}\\ 
\dot{\tempsigma}
\end{bmatrix}
 & {\tempIn} & \begin{bmatrix*}[l]
 {\tempfxi}_{\tempsigma}\tempxi +B_{\tempsigma} {\tempu}_c\\ 
 {\tempfchi}(\state,{\tempu}_c)\\ 
  0 \end{bmatrix*} = {\tempF}(\state,{\tempu}_c) 
 & \quad & (\state,{\tempu}_c)\in {\tempC}\\
\begin{bmatrix}{\tempxi}^+\\  
{\tempchi}^+\\  {\tempsigma}^+
\end{bmatrix} 
& \tempIn & \displaystyle\bigcup_{\tilde{\tempsigma}\in {\tempgsigma}(\state,{\tempu}_d)}
\begin{bmatrix*}[l]
\hat{\tempgxi}(\state,\tilde{\tempsigma},{\tempu}_d)\\ 
{\tempgchi} (\state,{\tempu}_d)\\ 
\tilde{\tempsigma}
\end{bmatrix*}=\tempdef \hat{\tempG}(\state,\tempu_d) 
 & \quad & (\state,{\tempu}_d)\in \hat{\tempD}\\
 \tempy_c & = & \temph_c(\state,{\tempu}_c) & & &\\
 \tempy_d & = & \temph_d(\state,{\tempu}_d) & & &
\end{array}
}
\right.
}

%% data of a TILDE hybrid DAE system with inputs and outputs, first line C, f or F, second line D, g or G. parameters
%% 
%% 1 state (x), 2 variable(xi),3 variable(chi),4 variable(sigma), 5 flow map 1(xi), 6 flow map 2(chi), 7 input(u), 8 output, 9 flow set, 10 \in or =, 11 flow map, 12 jump map 1(xi), 13 jump map 2(chi), 14 map 3(sigma), 15 jump set, 16 jump map, 17 output map, definition (:)

\newcommand{\tildehDAEInOutput}[9]
{
    \def\state{#1}%
    \def\tempxi{#2}%
    \def\tempchi{#3}%
    \def\tempsigma{#4}%
    \def\tempfxi{#5}%
    \def\tempfchi{#6}%
    \def\tempu{#7}%
    \def\tempy{#8}%
    \def\tempC{#9}%
    \tildehDAEInOutputcontinued
}
\newcommand{\tildehDAEInOutputcontinued}[9]
{
    \def\tempIn{#1}%10
    \def\tempF{#2}%11
    \def\tempgxi{#3}%12
    \def\tempgchi{#4}%13
    \def\tempgsigma{#5}%14
    \def\tempD{#6}%15
    \def\tempG{#7}%16
    \def\temph{#8}%17
    \def\tempdef{#9}%17    
%    \def\tempC{#9}%18
\left\{
{
%\setlength\extrarowheight{.2cm}
\begin{array}{r@{\ }c@{\ }lcr@{\ }c@{\ }l}
\begin{bmatrix}
\dot{\tempxi}\\ 
\dot{\tempchi}\\ 
\dot{\tempsigma}
\end{bmatrix}
 & {\tempIn} & \begin{bmatrix*}[l]
 \tilde{\tempfxi}_{\tempsigma}(\tempxi,{\tempu}_c)\\ 
 {\tempfchi}(\state,{\tempu}_c)\\ 
  0 \end{bmatrix*} =\tempdef \tilde{\tempF}(\state,{\tempu}_c) 
 & \quad & (\state,{\tempu}_c)\in {\tempC}\\
\begin{bmatrix}{\tempxi}^+\\  
{\tempchi}^+\\  {\tempsigma}^+
\end{bmatrix} 
& \tempIn & \displaystyle\bigcup_{\tilde{\tempsigma}\in {\tempgsigma}(\state,{\tempu}_d)}
\begin{bmatrix*}[l]
\hat{\tempgxi}(\state,\tilde{\tempsigma},{\tempu}_d)\\ 
{\tempgchi} (\state,{\tempu}_d)\\ 
\tilde{\tempsigma}
\end{bmatrix*}=\tempdef \hat{\tempG}(\state,\tempu_d) 
 & \quad & (\state,{\tempu}_d)\in \hat{\tempD}\\
 \tempy_c & = & \temph_c(\state,{\tempu}_c) & & &\\
 \tempy_d & = & \temph_d(\state,{\tempu}_d) & & &
\end{array}
}
\right.
}


%% data of a hybrid DAE system with inputs and outputs for switched DAE Arbitrary switching, first line C, f or F, second line D, g or G. parameters
%% 
%% 1 state (x), 2 variable(xi), 3 variable(sigma), 4 flow map 1(xi), 5 input(u), 6 output, 7 flow set, 8 \in or =, 9 flow map, 10 jump map 1(xi), 11 map 3(sigma), 12 jump set, 13 jump map, 14 output map, definition (:)

\newcommand{\hDAEIOshortSwDAE}[7]
{
    \def\state{#1}%
    \def\tempxi{#2}%
    \def\tempsigma{#3}%
    \def\tempfxi{#4}%
    \def\tempu{#5}%
    \def\tempy{#6}%
    \def\tempC{#7}%
    \hDAEIOshortSwDAEcontinued
}
\newcommand{\hDAEIOshortSwDAEcontinued}[8]
{
    \def\tempIn{#1}%8
    \def\tempF{#2}%9
    \def\tempgxi{#3}%10
    \def\tempgsigma{#4}%11
    \def\tempD{#5}%12
    \def\tempG{#6}%13
    \def\temph{#7}%14
    \def\tempdef{#8}%15    
%    \def\tempC{#9}%18
\left\{
{
\begin{array}{@{}r@{}@{}c@{}@{}l@{}@{}c@{}@{}cc@{}}
\begin{bmatrix}
E_{\tempsigma} & 0 \\ 
0&1
\end{bmatrix}
\begin{bmatrix}
\dot{\tempxi}\\ 
\dot{\tempsigma}
\end{bmatrix}
 & {=} &
 \begin{bmatrix*}[l]
 {\tempfxi}_{\tempsigma}\tempxi + B_{\tempsigma}{\tempu}_c\\ 
  0 \end{bmatrix*}  
 & \quad\quad & (\state,{\tempu}_c)\in {\tempC}\\
\begin{bmatrix}
{\tempxi}^+\\  
{\tempsigma}^+
\end{bmatrix} 
& \tempIn &
\displaystyle\bigcup_{\tilde{\tempsigma}\in {\tempgsigma}}
\begin{bmatrix*}[c]
{\tempgxi}\\ 
\tilde{\tempsigma}
\end{bmatrix*}
 & \quad\quad & (\state,{\tempu}_d)\in {\tempD}\\
 \tempy_c & = & \temph_{c,\tempsigma}(\tempxi,{\tempu}_c) & & &\\
 \tempy_d & = & \temph_{d,\tempsigma}(\tempxi,{\tempu}_d) & & &
\end{array}
}
\right.
}


%%% Highlight modifications
\def\startmodif{\color{blue}}
\def\stopmodif{\color{black}\normalcolor}



%%% Comments
%\newcommand{\rafal}[1]{{\noindent \color{red} \tt \small #1}}
%\newcommand{\andy}[1]{{\noindent \color{green} \tt \small #1}}
%\newcommand{\ricardo}[1]{{\noindent \color{blue} \tt \small #1}}
%\def\startmodif{\color{red}}
%\def\stopmodif{\color{black}\normalcolor}
%
%\def\startmodifa{\color{green}}
%\def\stopmodifa{\color{black}\normalcolor}
%
%\def\startmodifaa{\color{black}}
%\def\stopmodifaa{\color{black}\normalcolor}
%
%\def\startmodifr{\color{blue}}
%\def\stopmodifr{\color{black}\normalcolor}
%
%\def\startmodifra{\color{black}}
%\def\stopmodifra{\color{black}\normalcolor}


\newcommand{\Sphere}{\mathbb{S}}

\newcommand{\T}{\mathcal{T}}


\newcommand{\diff}{\tiny\mbox{diff}}

\newcommand{\imp}{\tiny\mbox{imp}}












