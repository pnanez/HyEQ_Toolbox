% This file was automatically created from the m-file 
% "m2tex.m" written by USL. 
% The fontencoding in this file is UTF-8. 
%  
% You will need to include the following two packages in 
% your LaTeX-Main-File. 
%  
% \usepackage{color} 
% \usepackage{fancyvrb} 
%  
% It is advised to use the following option for Inputenc 
% \usepackage[utf8]{inputenc} 
%  
  
% definition of matlab colors: 
\definecolor{mblue}{rgb}{0,0,1} 
\definecolor{mgreen}{rgb}{0.13333,0.5451,0.13333} 
\definecolor{mred}{rgb}{0.62745,0.12549,0.94118} 
\definecolor{mgrey}{rgb}{0.5,0.5,0.5} 
\definecolor{mdarkgrey}{rgb}{0.25,0.25,0.25} 
  
\DefineShortVerb[fontfamily=courier,fontseries=m]{\$} 
\DefineShortVerb[fontfamily=courier,fontseries=b]{\#} 
  
\begin{Verbatim}[commandchars=\$\{\},numbers=left,numbersep=2pt] 

    $textcolor{mblue}{function} [t j x] = HyEQsolver(f,g,C,D,x0,TSPAN,JSPAN,rule,options,solver,E) 
    $textcolor{mgreen}{%HYEQSOLVER solves hybrid equations.} 
    $textcolor{mgreen}{%   Syntax: [t j x] = HYEQSOLVER(f,g,C,D,x0,TSPAN,JSPAN,rule,options,solver,E)} 
    $textcolor{mgreen} 
    $textcolor{mgreen} 
    $textcolor{mgreen}{%   where x is the state, f is the flow map, g is the jump map, C is the} 
    $textcolor{mgreen}{%   flow set, and D is the jump set. It outputs the state trajectory (t,j)} 
    $textcolor{mgreen}{%   -> x(t,j), where t is the flow time parameter and j is the jump} 
    $textcolor{mgreen} 
    $textcolor{mgreen} 
    $textcolor{mgreen}{%   TSPAN = [TSTART TFINAL] is the time interval. JSPAN = [JSTART JSTOP] is} 
    $textcolor{mgreen}{%       the interval for discrete jumps. The algorithm stop when the first} 
    $textcolor{mgreen} 
    $textcolor{mgreen}{%   rule (optional parameter) - rule for jumps} 
    $textcolor{mgreen}{%       rule = 1 (default) -> priority for jumps rule = 2 -> priority for} 
    $textcolor{mgreen} 
    $textcolor{mgreen}{%   options (optional parameter) - options for the solver see odeset f.ex.} 
    $textcolor{mgreen}{%       options = odeset('RelTol',1e-6);} 
    $textcolor{mgreen} 
    $textcolor{mgreen}{%   solver (optional parameter. String) - selection of the desired ode} 
    $textcolor{mgreen}{%       solver. All ode solvers are suported, exept for ode15i.  See help} 
    $textcolor{mgreen} 
    $textcolor{mgreen}{%   E (optional parameter) - Mass matrix [constant matrix | function_handle]} 
    $textcolor{mgreen}{%       For problems: } 
    $textcolor{mgreen}{%       E*\dot{x} = f(x) x \in C } 
    $textcolor{mgreen}{%       x^+ = g(x)  x \in D} 
    $textcolor{mgreen}{%       set this property to the value of the constant mass matrix. For} 
    $textcolor{mgreen}{%       problems with time- or state-dependent mass matrices, set this} 
    $textcolor{mgreen}{%       property to a function that evaluates the mass matrix. See help} 
    $textcolor{mgreen} 
    $textcolor{mgreen} 
    $textcolor{mgreen}{%         % Consider the hybrid system model for the bouncing ball with data given in} 
    $textcolor{mgreen}{%         % Example 1.2. For this example, we consider the ball to be bouncing on a} 
    $textcolor{mgreen}{%         % floor at zero height. The constants for the bouncing ball system are} 
    $textcolor{mgreen}{%         % \gamma=9.81 and \lambda=0.8. The following procedure is used to} 
    $textcolor{mgreen}{%         % simulate this example in the Lite HyEQ Solver:} 
    $textcolor{mgreen}{%} 
    $textcolor{mgreen}{%         % * Inside the MATLAB script run_ex1_2.m, initial conditions, simulation} 
    $textcolor{mgreen}{%         % horizons, a rule for jumps, ode solver options, and a step size} 
    $textcolor{mgreen}{%         % coefficient are defined. The function HYEQSOLVER.m is called in order to} 
    $textcolor{mgreen}{%         % run the simulation, and a script for plotting solutions is included.} 
    $textcolor{mgreen}{%         % * Then the MATLAB functions f_ex1_2.m, C_ex1_2.m, g_ex1_2.m, D_ex1_2.m} 
    $textcolor{mgreen}{%         % are edited according to the data given below.} 
    $textcolor{mgreen}{%         % * Finally, the simulation is run by clicking the run button in} 
    $textcolor{mgreen}{%         % run_ex1_2.m or by calling run_ex1_2.m in the MATLAB command window.} 
    $textcolor{mgreen}{%} 
    $textcolor{mgreen}{%         % For further information, type in the command window:} 
    $textcolor{mgreen} 
    $textcolor{mgreen}{%         % Define initial conditions} 
    $textcolor{mgreen}{%         x1_0 = 1;} 
    $textcolor{mgreen}{%         x2_0 = 0;} 
    $textcolor{mgreen} 
    $textcolor{mgreen}{%         % Set simulation horizon} 
    $textcolor{mgreen}{%         TSPAN = [0 10];} 
    $textcolor{mgreen} 
    $textcolor{mgreen}{%         % Set rule for jumps and ODE solver options} 
    $textcolor{mgreen} 
    $textcolor{mgreen}{%         % rule = 1 -> priority for jumps} 
    $textcolor{mgreen} 
    $textcolor{mgreen}{%         % rule = 2 -> priority for flows} 
    $textcolor{mgreen} 
    $textcolor{mgreen}{%         % set the maximum step length. At each run of the} 
    $textcolor{mgreen}{%         % integrator the option 'MaxStep' is set to} 
    $textcolor{mgreen}{%         % (time length of last integration)*maxStepCoefficient.} 
    $textcolor{mgreen}{%         %  Default value = 0.1} 
    $textcolor{mgreen}{%} 
    $textcolor{mgreen} 
    $textcolor{mgreen} 
    $textcolor{mgreen}{%         % Simulate using the HYEQSOLVER script} 
    $textcolor{mgreen}{%         % Given the matlab functions that models the flow map, jump map,} 
    $textcolor{mgreen}{%         % flow set and jump set (f_ex1_2, g_ex1_2, C_ex1_2, and D_ex1_2} 
    $textcolor{mgreen}{%         % respectively)} 
    $textcolor{mgreen}{%} 
    $textcolor{mgreen}{%         [t j x] = HYEQSOLVER( @f_ex1_2,@g_ex1_2,@C_ex1_2,@D_ex1_2,...} 
    $textcolor{mgreen} 
    $textcolor{mgreen}{%         % plot solution} 
    $textcolor{mgreen}{%} 
    $textcolor{mgreen}{%         figure(1) % position} 
    $textcolor{mgreen}{%         clf} 
    $textcolor{mgreen}{%         subplot(2,1,1),plotflows(t,j,x(:,1))} 
    $textcolor{mgreen}{%         grid on} 
    $textcolor{mgreen} 
    $textcolor{mgreen}{%         subplot(2,1,2),plotjumps(t,j,x(:,1))} 
    $textcolor{mgreen}{%         grid on} 
    $textcolor{mgreen} 
    $textcolor{mgreen}{%         figure(2) % velocity} 
    $textcolor{mgreen}{%         clf} 
    $textcolor{mgreen}{%         subplot(2,1,1),plotflows(t,j,x(:,2))} 
    $textcolor{mgreen}{%         grid on} 
    $textcolor{mgreen} 
    $textcolor{mgreen}{%         subplot(2,1,2),plotjumps(t,j,x(:,2))} 
    $textcolor{mgreen}{%         grid on} 
    $textcolor{mgreen} 
    $textcolor{mgreen}{%         % plot hybrid arc} 
    $textcolor{mgreen}{%         } 
    $textcolor{mgreen}{%         figure(3)} 
    $textcolor{mgreen}{%         plotHybridArc(t,j,x)} 
    $textcolor{mgreen}{%         xlabel('j')} 
    $textcolor{mgreen}{%         ylabel('t')} 
    $textcolor{mgreen} 
    $textcolor{mgreen}{%         % plot solution using plotHarc and plotHarcColor} 
    $textcolor{mgreen}{%} 
    $textcolor{mgreen}{%         figure(4) % position} 
    $textcolor{mgreen}{%         clf} 
    $textcolor{mgreen}{%         subplot(2,1,1), plotHarc(t,j,x(:,1));} 
    $textcolor{mgreen}{%         grid on} 
    $textcolor{mgreen}{%         ylabel('x_1 position')} 
    $textcolor{mgreen}{%         subplot(2,1,2), plotHarc(t,j,x(:,2));} 
    $textcolor{mgreen}{%         grid on} 
    $textcolor{mgreen} 
    $textcolor{mgreen}{%} 
    $textcolor{mgreen}{%         % plot a phase plane} 
    $textcolor{mgreen}{%         figure(5) % position} 
    $textcolor{mgreen}{%         clf} 
    $textcolor{mgreen}{%         plotHarcColor(x(:,1),j,x(:,2),t);} 
    $textcolor{mgreen}{%         xlabel('x_1')} 
    $textcolor{mgreen}{%         ylabel('x_2')} 
    $textcolor{mgreen} 
    $textcolor{mgreen}{%--------------------------------------------------------------------------} 
    $textcolor{mgreen}{% Matlab M-file Project: HyEQ Toolbox @  Hybrid Systems Laboratory (HSL),} 
    $textcolor{mgreen}{% https://hybrid.soe.ucsc.edu/software} 
    $textcolor{mgreen}{% http://hybridsimulator.wordpress.com/} 
    $textcolor{mgreen}{% Filename: HYEQSOLVER.m} 
    $textcolor{mgreen}{%--------------------------------------------------------------------------} 
    $textcolor{mgreen}{%   See also HYEQSOLVER, PLOTARC, PLOTARC3, PLOTFLOWS, PLOTHARC,} 
    $textcolor{mgreen}{%   PLOTHARCCOLOR, PLOTHARCCOLOR3D, PLOTHYBRIDARC, PLOTJUMPS.} 
    $textcolor{mgreen}{%   Copyright @ Hybrid Systems Laboratory (HSL),} 
    $textcolor{mgreen}{%   Revision: 0.0.0.4 Date: 04/6/2017 16:26:00} 
     
     
    $textcolor{mblue}{if} ~exist($textcolor{mred}{'rule'},$textcolor{mred}{'var'}) 
        rule = 1; 
    $textcolor{mblue}{end} 
     
    $textcolor{mblue}{if} ~exist($textcolor{mred}{'options'},$textcolor{mred}{'var'}) 
        options = odeset(); 
    $textcolor{mblue}{end} 
    $textcolor{mblue}{if} exist($textcolor{mred}{'E'},$textcolor{mred}{$textcolor{mred}{'var'}}) && ~exist($textcolor{mred}{'solver'},$textcolor{mred}{$textcolor{mred}{'var'}}) 
        solver = $textcolor{mred}{'ode15s'}; 
    $textcolor{mblue}{end} 
    $textcolor{mblue}{if} ~exist($textcolor{mred}{'solver'},$textcolor{mred}{'var'}) 
        solver = $textcolor{mred}{'ode45'}; 
    $textcolor{mblue}{end} 
    $textcolor{mblue}{if} ~exist($textcolor{mred}{'E'},$textcolor{mred}{'var'}) 
        E = []; 
    $textcolor{mblue}{end} 
    $textcolor{mgreen}{% mass matrix (if existent)} 
    isDAE = false; 
    $textcolor{mblue}{if} ~isempty(E) 
        isDAE = true; 
        $textcolor{mblue}{switch} isa(E,$textcolor{mred}{'function_handle'}) 
            $textcolor{mblue}{case} true $textcolor{mgreen}{% Function E(x)} 
                M = E; 
                options = odeset(options,$textcolor{mred}{'Mass'},M,$textcolor{mred}{'Stats'},$textcolor{mred}{'off'},... 
                    $textcolor{mred}{'MassSingular'},$textcolor{mred}{'maybe'},$textcolor{mred}{'MStateDependence'},$textcolor{mred}{'strong'},... 
                    $textcolor{mred}{'InitialSlope'},f_hdae(x0,TSPAN(1)));  
            $textcolor{mblue}{case} false $textcolor{mgreen}{% Constant double matrix} 
                M = double(E); 
                options = odeset(options,$textcolor{mred}{'Mass'},M,$textcolor{mred}{'Stats'},$textcolor{mred}{'off'},... 
                    $textcolor{mred}{'MassSingular'},$textcolor{mred}{'maybe'},$textcolor{mred}{'MStateDependence'},$textcolor{mred}{'none'}); 
        $textcolor{mblue}{end} 
    $textcolor{mblue}{end} 
     
    odeX = str2func(solver); 
    nargf = nargin(f); 
    nargg = nargin(g); 
    nargC = nargin(C); 
    nargD = nargin(D); 
     
     
     
    $textcolor{mgreen}{% simulation horizon} 
    tstart = TSPAN(1); 
    tfinal = TSPAN(end); 
    jout = JSPAN(1); 
    j = jout(end); 
     
    $textcolor{mgreen}{% simulate} 
    tout = tstart; 
    [rx,cx] = size(x0); 
    $textcolor{mblue}{if} rx == 1 
        xout = x0; 
    $textcolor{mblue}{elseif} cx == 1 
        xout = x0.'; 
    $textcolor{mblue}{else} 
        error($textcolor{mred}{'Error, x0 does not have the proper size'}) 
    $textcolor{mblue}{end} 
     
    $textcolor{mgreen}{% Jump if jump is prioritized:} 
    $textcolor{mblue}{if} rule == 1 
        $textcolor{mblue}{while} (j<JSPAN(end)) 
            $textcolor{mgreen}{% Check if value it is possible to jump current position} 
            insideD = fun_wrap(xout(end,:).',tout(end),j,D,nargD); 
            $textcolor{mblue}{if} insideD == 1 
                [j $textcolor{mred}{tout jout xout] = jump(g,j,tout,jout,xout,nargg);} 
            $textcolor{mblue}{else} 
                break; 
            $textcolor{mblue}{end} 
        $textcolor{mblue}{end} 
    $textcolor{mblue}{end} 
    fprintf($textcolor{mred}{'Completed: %3.0f%%'},0); 
    $textcolor{mblue}{while} (j < JSPAN(end) && tout(end) < TSPAN(end)) 
        options = odeset(options,$textcolor{mred}{'Events'},@(t,x) zeroevents(x,t,j,C,D,... 
            rule,nargC,nargD)); 
        $textcolor{mgreen}{% Check if it is possible to flow from current position} 
        insideC = fun_wrap(xout(end,:).',tout(end),j,C,nargC); 
        $textcolor{mblue}{if} insideC == 1 
            $textcolor{mblue}{if} isDAE 
                options = odeset(options,$textcolor{mred}{'InitialSlope'},f(xout(end,:).',tout(end))); 
            $textcolor{mblue}{end} 
            [t,x] = odeX(@(t,x) fun_wrap(x,t,j,f,nargf),[tout(end) tfinal],... 
                xout(end,:).', $textcolor{mred}{options);} 
            nt = length(t); 
            tout = [tout; t]; 
            xout = [xout; x]; 
            jout = [jout; j*ones(1,nt)']; 
        $textcolor{mblue}{end} 
         
        $textcolor{mgreen}{%Check if it is possible to jump} 
        insideD = fun_wrap(xout(end,:).',tout(end),j,D,nargD); 
        $textcolor{mblue}{if} insideD == 0 
            break; 
        $textcolor{mblue}{else} 
            $textcolor{mblue}{if} rule == 1 
                $textcolor{mblue}{while} (j<JSPAN(end)) 
                    $textcolor{mgreen}{% Check if it is possible to jump from current position} 
                    insideD = fun_wrap(xout(end,:).',tout(end),j,D,nargD); 
                    $textcolor{mblue}{if} insideD == 1 
                        [j $textcolor{mred}{tout jout xout] = jump(g,j,tout,jout,xout,nargg);} 
                    $textcolor{mblue}{else} 
                        break; 
                    $textcolor{mblue}{end} 
                $textcolor{mblue}{end} 
            $textcolor{mblue}{else} 
                [j $textcolor{mred}{tout jout xout] = jump(g,j,tout,jout,xout,nargg);} 
            $textcolor{mblue}{end} 
        $textcolor{mblue}{end} 
        fprintf($textcolor{mred}{'\b\b\b\b%3.0f%%'},max(100*j/JSPAN(end),100*tout(end)/TSPAN(end))); 
    $textcolor{mblue}{end} 
    t = tout; 
    x = xout; 
    j = jout; 
    fprintf($textcolor{mred}{'\nDone\n'}); 
    $textcolor{mblue}{end} 
      
\end{Verbatim}  
  
\UndefineShortVerb{\$} 
\UndefineShortVerb{\#} 
 